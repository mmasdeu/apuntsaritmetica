\subsection{Com factoritza un polinomi mòdul diferents primers, variació 1}
 
 En aquest projecte s'estudia la factorització d'un polinomi irreductible amb coeficients enters, mòdul diferents primers. Donat un polinomi $f(x)\in\Z[x]$, definim el seu tipus de factorització mòdul $p$ de la manera següent: suposem que la imatge $\bar f(x)\in \F_p[x]$ factoritza com
 \[
 \bar f(x)= \bar f_1(x)^{e_1}\cdots \bar f_g(x)^{e_g}\in \F_p[x],
 \]
 on $\deg \bar f_i(x)= d_i$. Aleshores direm que $f$ té tipus de factorització $(d_1^{e_1},\ldots,d_g^{e_g})$ (on l'ordre no importa). Per exemple, un polinomi de grau $2$ pot tenir tipus $(2)$, $(1^2)$ o $(1,1)$ (no posem els exponents si són $1$).

 Escriviu un programa que, donat un polinomi irreductible, calculi el tipus de factorització mòdul tots els primers fins una fita donada. Direm que un enter positiu $M$ és el \emph{mòdul} d'aquest polinomi si el tipus de factorització mòdul $p$ depèn de la classe de $p$ mòdul $M$, i $M$ és el menor enter amb aquesta propietat.
 
 Comenceu estudiant polinomis quadràtics, i vegeu quins tipus apareixen. Calculeu el mòdul de qualsevol polinomi quadràtic. Això s'hauria de poder demostrar, fent servir el què hem vist a classe.
 
 Després continueu estudiant els polinomis cúbics. Escriviu un programa que intenti trobar el mòdul d'un polinomi. Podeu trobar exemples de polinomis cúbics amb mòdul i d'altres sense?.
 
 Seguiu amb polinomis de grau més alt. Podeu estudiar els polinomis ciclotòmics (tenen mòdul? quins tipus de factorització apareixen?).
 
 Donat un polinomi irreductible, sabrieu predir si té mòdul o no, a partir d'informació que podeu trobar per exemple a \url{lmfdb.org}?
 
 \subsection{Com factoritza un polinomi mòdul diferents primers, variació 2}
 
 En aquest projecte s'estudia la factorització d'un polinomi irreductible amb coeficients enters, mòdul diferents primers. Donat un polinomi $f(x)\in\Z[x]$, definim el seu tipus de factorització mòdul $p$ de la manera següent: suposem que la imatge $\bar f(x)\in \F_p[x]$ factoritza com
 \[
 \bar f(x)= \bar f_1(x)^{e_1}\cdots \bar f_g(x)^{e_g}\in \F_p[x],
 \]
 on $\deg \bar f_i(x)= d_i$. Aleshores direm que $f$ té tipus de factorització $(d_1^{e_1},\ldots,d_g^{e_g})$ (on l'ordre no importa). Per exemple, un polinomi de grau $2$ pot tenir tipus $(2)$, $(1^2)$ o $(1,1)$ (no posem els exponents si són $1$).
 
 Escriviu un programa que, donat un polinomi irreductible, calculi el tipus de factorització mòdul tots els primers fins una fita donada.
 
 Ens interessa veure com es reparteixen els primers entre els diferents tipus de factorització. Observeu quina proporció de primers hi ha en cadascuna de les tres classes per polinomis de grau $2$.
 
 Després continueu estudiant els polinomis cúbics. Hauríeu de veure diferent comportament depenent del polinomi que trieu.
 
 Seguiu amb polinomis de grau més alt. Intenteu predir quines proporcions trobareu, a partir d'informació que trobeu a \url{lmfdb.org}.
 

 \subsection{Nombre de punts mòdul \texorpdfstring{$p$}{p} per corbes el·líptiques: variació I}
 Estudieu la distribució de $\#E(\F_p)$, i després la de
 \[
 \frac{p+1 - \#E(\F_p)}{2\sqrt{p}}
 \]
 per diverses corbes el·líptiques (donada una corba el·líptica $E$ en \texttt{Sage}, el mètode \texttt{E.ap(p)} us retorna $p+1-\#E(\F_p)$ de manera bastant ràpida). Dibuixeu histogrames dels resultats, i intenteu identificar les distribucions resultants. Compareu en particular el resultat que obteniu per aquestes dues corbes:
 \begin{align*}
     E_0\colon & y^2 + y = x^3 - x^2,\\
     E_1\colon & y^2 + y = x^3.
 \end{align*}

 Podeu fer servir la funció \texttt{histogram()} de \texttt{Sage} per estudiar les distribucions. Trobeu unes quantes corbes que es comportin com $E_0$ i unes quantes que es comportin com $E_1$. Quina propietat d'una corba el·líptica (de les que trobeu a \url{lmfdb.org}) fa que es comporti d'una manera o una altra?
 
 \subsection{Nombre de punts mòdul \texorpdfstring{$p$}{p} per corbes el·líptiques: variació II}
 Donada una corba el·líptica $E$ definida sobre $\Q$, definim la funció
 \[
 C_E(x) = \prod_{p\leq x}\frac{\#E(\F_p)}{p},\quad x\in\R_{>0}.
 \]
 Estudieu el comportament assimptòtic de $C_E$ ($x\to\infty$) per diverses corbes el·líptiques. Per exemple, considereu les corbes:
 \begin{align*}
E_0\colon& y^2+y=x^3-x^2,\\
E_1\colon& y^2+y=x^3-x,\\
E_2\colon& y^2+y = x^3+x^2-2x,\\
E_3\colon&y^2 +y =x^3 - 7x+6.

\end{align*}
Haurieu de veure un comportament molt diferenciat entre $E_0$ i la resta d'exemples. Intenteu comparar $C_E(x)$ amb $K\log(x)^n$ per alguna constant $K$ i algun enter $n\geq 0$. Què pot ser que $n$ ens digui sobre la corba $E$? Haurieu de poder trobar la resposta consultant la web \url{lmfdb.org}.

