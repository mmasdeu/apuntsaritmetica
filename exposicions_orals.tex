\begin{enumerate}
\item Caracteritzar quins nombres es poden expressar com a suma de dos quadrats.

  {\tiny Aigner, M. and Ziegler, G. ``Proofs from The Book''}


\item Demostracions ``Euclidianes'' de l'existència d'infinits primers en algunes successions aritmètiques.

  {\tiny Prime Numbers in certian arithmetic progressions. Ram Murty and Nithum Thain}

\item El teorema de Schur: $\{p\text{ primer tal que } p \mid f(n)\neq 0\}$ és infinit, per qualsevol polinomi $f(x)$ amb coeficients enters.

  {\tiny Aigner, M. and Ziegler, G. ``Proofs from The Book''}
\item El postulat de Bertrand: per a tot $n\geq 1$, $[n,2n]\cap \{\text{primers}\}\neq\emptyset$.

  {\tiny Aigner, M. and Ziegler, G. ``Proofs from The Book''}
\item El teorema de Mills: $\exists A$ tal que $\lfloor A^{3^n}\rfloor$ és primer per a tot $n$.

  {\tiny Mills, W. H. "A Prime-Representing Function." Bull. Amer. Math. Soc. 53, 604, 1947.}
\item La irracionalitat de $\pi$ fent servir fraccions continuades.

  {\tiny Laczkovich, M. ``On Lambert's Proof of the Irrationality of $\pi$''}
 \item La irracionalitat de $e$ i d'$e^2+re$ per a tot $r\in\Q$ fent servir fraccions continuades.

   {\tiny Olds, C. ``The Simple Continued Fraction Expansion of $e$''}
 \item L'equació de Pell, i la seva solució amb fraccions continuades.

   {\tiny \url{https://www.math.ubc.ca/~gor/Math323_2013/pell.pdf}}
\item La sèrie $\displaystyle\sum_{p\text{ primer}} \frac 1 p$ és divergent.

  {\tiny Aigner, M. and Ziegler, G. ``Proofs from The Book''}

\item El teorema dels nombres primers (sense demostració).

  {\tiny Apostol, T. ``Introduction to Analytic Number Theory'', Chapters 4 and 13}

\item El ``primer cas'' de l'Últim Teorema de Fermat: si $p$ és regular i $x^p+y^p=z^p$, aleshores $p\mid xyz$.

  {\tiny Marcus, D. ``Number Fields''}

\item Els nombres $p$-àdics: definició, i el lema de Hensel.

  {\tiny Gouvea, F. ``$p$-adic numbers: An Introduction''}

\item El principi de Hasse local-global: teorema de Minkowski.

  {\tiny Serre, J.-P. ``A course in arithmetic''}

  \item EL principi de Hasse local-global: l'exemple de Selmer.

    {\tiny \url{https://kconrad.math.uconn.edu/blurbs/gradnumthy/selmerexample.pdf}}

  \item Teorema d'Schnirelmann: si $A$ té densitat (d'Schnirelmann) positiva, existeix una $k$ tal que tot nombre natural és suma de com a molt $k$ nombres de $A$.

    {\tiny \url{http://jonismathnotes.blogspot.com/2015/04/schnirelmann-density-and-warings-problem.html}}
  \item Prime gaps, la història.

    {\tiny \url{https://dms.umontreal.ca/~andrew/CurrentEventsArticle.pdf}}
  \item La conjectura de Goldbach, i remarques sobre la demostració de la versió dèbil (ternària).

    {\tiny Helfgott, H. ``The ternary Goldbach conjecture is true''}
  \item Conjunts de Sidon, i la construcció de Ruzsa.

    {\tiny Rusza, I. ``Solving a linear equation in a set of integers I''}
\end{enumerate}