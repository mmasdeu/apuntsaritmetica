
\begin{enumerate}
%\setlength{\itemsep}{-2pt}

\item Demostracions ``Euclidianes'' de l'existència d'infinits primers en successions aritmètiques.

  {\tiny Prime Numbers in certian arithmetic progressions. Ram Murty and Nithum Thain}

\item El teorema de Schur: $\{p\text{ primer tal que } p \mid f(n)\neq 0\}$ és infinit, per a tot $f(x)\in\Z[x]$.

  {\tiny Aigner, M. and Ziegler, G. ``Proofs from The Book''}
%\item El postulat de Bertrand: per a tot $n\geq 1$, $[n,2n]\cap \{\text{primers}\}\neq\emptyset$.

%  {\tiny Aigner, M. and Ziegler, G. ``Proofs from The Book''}
\item El teorema de Mills: $\exists A$ tal que $\lfloor A^{3^n}\rfloor$ és primer per a tot $n$.

  {\tiny Mills, W. H. "A Prime-Representing Function." Bull. Amer. Math. Soc. 53, 604, 1947.}
\item La irracionalitat de $\pi$ fent servir fraccions continuades.

  {\tiny Laczkovich, M. ``On Lambert's Proof of the Irrationality of $\pi$''}
% \item La irracionalitat de $e$ i d'$e^2+re$ per a tot $r\in\Q$ fent servir fraccions continuades.

%   {\tiny Olds, C. ``The Simple Continued Fraction Expansion of $e$''}
 \item El teorema d'aproximació de Dirichlet.

   {\tiny \url{https://www.math.uzh.ch/gorodnik/nt/lecture12.pdf} (primera pàgina)}
   
 \item La solució de l'equació de Pell a partir del teorema de Dirichlet.
 
 {\tiny \url{http://www-personal.umich.edu/~hlm/math475/pell.pdf}}
 
\item La suma dels recíprocs dels nombres primers és divergent.

  {\tiny Aigner, M. and Ziegler, G. ``Proofs from The Book''}

\item El teorema dels nombres primers (sense demostració).

  {\tiny Apostol, T. ``Introduction to Analytic Number Theory'', Chapters 4 and 13}

\item El ``primer cas'' de l'Últim Teorema de Fermat: $p$ és regular i $x^p+y^p=z^p\implies p\mid xyz$.

  {\tiny Marcus, D. ``Number Fields''}

\item Els nombres $p$-àdics: definició, i el lema de Hensel.

  {\tiny Gouvea, F. ``$p$-adic numbers: An Introduction''}

\item El principi de Hasse local-global: teorema de Minkowski.

  {\tiny Serre, J.-P. ``A course in arithmetic''}

  \item El principi de Hasse local-global: l'exemple de Selmer.

    {\tiny \url{https://kconrad.math.uconn.edu/blurbs/gradnumthy/selmerexample.pdf}}

  \item El Teorema d'Schnirelmann.

    {\tiny \url{http://jonismathnotes.blogspot.com/2015/04/schnirelmann-density-and-warings-problem.html}}
    
  \item Prime gaps, la història.

    {\tiny \url{https://dms.umontreal.ca/~andrew/CurrentEventsArticle.pdf}}
    
  \item La conjectura de Goldbach, i remarques sobre la demostració de la versió dèbil (ternària).

    {\tiny Helfgott, H. ``The ternary Goldbach conjecture is true''}
    
  \item Conjunts de Sidon, i la construcció de Ruzsa.

    {\tiny Rusza, I. ``Solving a linear equation in a set of integers I''}

\item Carreres de primers.

{\tiny \url{https://dms.umontreal.ca/~andrew/PDF/PrimeRace.pdf}}

\item L'enunciat de la conjectura ABC, i una justificació heurística.

{\tiny https://terrytao.wordpress.com/2012/09/18/the-probabilistic-heuristic-justification-of-the-abc-conjecture/}

\item La conjectura ABC implica Fermat assimptòtic.

{\tiny \url{http://www.ams.org/notices/200210/fea-granville.pdf}}

\item L'altura canònica en corbes el·líptiques i el regulador.

{\tiny Silverman, J. ``The arithmetic of elliptic curves'',  capítol VIII.9}

\item Criptografia basada en isogènies.

{\tiny \url{https://eprint.iacr.org/2011/506.pdf}}

\item Signatura digital amb RSA.

{\tiny \url{http://cacr.uwaterloo.ca/hac/about/chap11.pdf}, pg.~433}


\item Xifrat (parcialment) homomòrfic: el xifrat de Pailler.

{\tiny \url{https://en.wikipedia.org/wiki/Paillier_cryptosystem}}

\item L'equació de S-unitats i el teorema de Siegel.

{\tiny Hindry, M. and Silverman, J. ``Diophantine Approximation'',  Theorem D.8.1}

\item La multiplicació de Karatsuba, anàlisi de la complexitat.

{\tiny \url{https://en.wikipedia.org/wiki/Karatsuba_algorithm}
}

\item El teorema de Cayley--Bacharach. Aplicació a la llei de grup d'una corba el·líptica.

{\tiny \url{https://staff.math.su.se/shapiro/UIUC/DingPlaneCurves.pdf}
}

\item La funció L d'una corba el·líptica, i la conjectura de Birch i Swinnerton-Dyer.

{\tiny \url{https://www.claymath.org/sites/default/files/birchswin.pdf}
}

\item L'enunciat del teorema de modularitat de corbes el·líptiques.

{\tiny \url{http://www.ams.org/notices/199911/comm-darmon.pdf}}

\item La demostració del teorema de Nagell--Lutz.

{\tiny Silverman, J., Tate, J. ``Rational Points on Elliptic Curves''}

\item Teorema de Szemerédi, Teorema de Green--Tao i la conjectura d'Erdös sobre progressions aritmètiques.

{\tiny \url{https://en.wikipedia.org/wiki/Szemer%C3%A9di%27s_theorem}}

\item Sistemes recobridors: definicions i algun exemple (amb demostració)

{\tiny \url{http://maths.nju.edu.cn/~zwsun/Cover.pdf}}

\item El teorema dels 6 exponencials i la conjectura dels 4 exponencials: enunciats i alguna conseqüència.

{\tiny \url{https://en.wikipedia.org/wiki/Six_exponentials_theorem}}

\end{enumerate}