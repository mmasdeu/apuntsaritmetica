\begin{enumerate}
    \item Caracteritzar quins primers es poden expressar com a suma de dos quadrats.
    \item Demostracions ``Euclidianes'' de l'existència d'infinits primers en algunes successions aritmètiques.
    \item Demostració topològica de la infinitud dels primers (Fustenberg).
    \item El postulat de Bertrand: per a tot $n\geq 1$, $[n,2n]\cap \{\text{primers}\}\neq\emptyset$.
    \item El teorema de Mill: $\exists A$ tal que $\lfloor A^{3^n}\rfloor$ és primer per a tot $n$.
    \item La irracionalitat de $\pi$ fent servir fraccions contínues (Lakzkovich).
    \item La irracionalitat de $e$ i d'$e^2+re$ per a tot $r\in\Q$ fent servir fraccions contínues.
    \item L'equació de Pell, i la seva solució amb fraccions contínues.
    \item L'estructura de grup de $(\Z/m\Z)^\times$.
    \item La sèrie $\displaystyle\sum_{p\text{ primer}} \frac 1 p$ és divergent.
    \item El teorema dels nombres primers (sense demostració).
    \item El ``primer cas'' de l'Últim Teorema de Fermat.
    \item Els nombres p-àdics: definició, el lema de Hensel.
    \item El principi de Hasse.
    \item Teorema d'Schnirelmann: si A té densitat (d'Schnirelmann) positiva, existeix una $k$ tal que tot nombre natural és suma de com a molt $k$ nombres de A.
    \item Prime gaps, la història.
    \item La conjectura de Goldbach, i remarques sobre la demostració versió dèbil (ternària).
    \item Conjunts de Sidon, i la construcció de Ruzsa.
\end{enumerate}