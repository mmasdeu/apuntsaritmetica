\subsection{Divisibilitat}
\begin{theorem}[Divisió entera]
Donats enters $a$ i $b$ amb $b > 1$, existeixen enters únics $q$ i $r$ tals que
\[
a = bq +r,\quad 0\leq r < d.
\]
L'enter $q$ s'anomena el quocient d'$a$ entre $b$, i $r$ s'anomena el residu.
\end{theorem}
En general, diem que $a$ divideix $b$ si existeix un enter $q$ tal que $aq=b$. Escriurem $a\mid b$.

Una primera aplicació és el fet que qualsevol enter admet representacions en qualsevol base:
\begin{theorem}[representació $m$-àdica o en base $m$]
Sigui $m\geq 2$. Aleshores tot enter positiu $n$ es pot escriure de manera única com
\[
n = a_0 + a_1 m+a_2m^2+\cdots+ a_km^k, \quad 0\leq a_i\leq m-1,\quad a_k\neq 0.
\]
on $k$ és l'únic enter que satisfà
\[
m^k \leq n < m^{k+1}.
\]
\end{theorem}

Passem ara a parlar del màxim comú divisor (que escriurem $\gcd$, de l'anglès \emph{greatest common divisor}). El màxim comú divisor dels nombres $a$ i $b$ es defineix com
\[
\gcd(a,b)=\max\{d ~\colon~ d\mid a\text{ i } d\mid b\}.
\]
També definim $\gcd(0,0)=0$.

\begin{lemma}
Es té:
\[
\gcd(a,b) = \gcd(b,a)=\gcd(\pm a,\pm b) = \gcd(a,b-a) = \gcd(a,b+a). 
\]
\end{lemma}
\begin{proof}
   Directament de la definició i el fet que si $d\mid a$ i $d\mid b$ aleshores $d\mid a\pm b$.
\end{proof}

Observem que, com a conseqüència, també obtenim:
\begin{corollary}
Per a tot $a$, $b$ i $t \in \Z$ es té
\[
\gcd(a,b+at) = \gcd(a,b).
\]
\end{corollary}

Aquest últim resultat ens permet calcular el màxim comú divisor entre dos nombres de manera ràpida. Comencem amb un exemple:

\begin{example}
Calculem $\gcd(986,289)$. Fent la divisió entera, obtenim
\[
986 = 3\cdot 289 + 119,
\]
i per tant
\[
\gcd(986,289) = \gcd(3\cdot 289+119,289) = \gcd(119,289).
\]
Seguim ara amb una nova divisió:
\[
289 = 2\cdot 119 + 51,
\]
que ens dona
\[
\gcd(119,289) = \gcd(119,2\cdot 119 + 51) = \gcd(119, 51).
\]
Seguim amb
\[
119 = 2\cdot 51 + 17,
\]
i per tant
\[
\gcd(119,51) = \gcd(2\cdot 51+ 17,51) = \gcd(17,51).
\]
Finalment, com que $51=17\cdot 3$, obtenim $\gcd(17,51) = 17$.
\end{example}

Aquest procediment es pot escriure en forma d'algoritme:

\begin{algo}

\begin{python}
def gcd(a,b):
    while b:
        a, b = b, a % b
    return a.abs()
  \end{python}
\caption{Calcula el màxim comú divisor de dos enters $a$ i $b$.}
\end{algo}

També és fàcil de veure que
\begin{lemma}
Per a tot $a,b,n\in\Z$ es té:
\[
\gcd(an,bn) = |n|\gcd(a,b).
\]
\end{lemma}
\begin{proof}
Podem assumir (canviant signes i reordenant, si cal) que $a\geq b\geq 1$ i $n>0$. Farem la demostració per inducció sobre $a+b\geq 2$. El cas base és $a=b=1$ i és obvi. Per fer el cas general, escrivim
\[
a=bq+r,\quad 0\leq r<b,
\]
i aleshores
\[
an = bnq + rn,
\]
per tant:
\[
\gcd(an,bn) = \gcd(bnq+rn,bn)=\gcd(rn,bn)=|n|\gcd(r,b)=|n|\gcd(a,b),
\]
on a la tercera igualtat hem aplicat la hipòtesi d'inducció (com que $r < b\leq a$, tenim $r+b < a + b$).
\end{proof}

Finalment, també és important veure que el $\gcd$ satisfà una maximalitat més forta que la que diu explícitament la seva definició:
\begin{lemma}
Siguin $a,b,n\in\Z$ i suposem que $n\mid a$ i $n\mid b$. Aleshores $n\mid\gcd(a,b)$.
\end{lemma}
\begin{proof}
 Escrivim $a=na'$ i $b=nb'$. Per tant,
 \[
 \gcd(a,b)=\gcd(na',nb') = n\gcd(a',b')
 \]
 i veiem que $n$ divideix $\gcd(a,b)$.
\end{proof}

\subsection{Factorització d'enters}
Recordem que un primer és un nombre positiu $p$ que té exactament dos divisors positius ($1$ i $p$). L'objectiu d'aquesta subsecció és demostrar el següent resultat, que ens diu que els nombres primers són els ``blocs'' amb els quals es construeixen tots els nombres naturals. 
\begin{theorem}[Teorema fonamental de l'aritmètica]
\label{thm:tfa}
Tot enter positiu $n$ es pot escriure com a producte de primers:
\[
n=p_1 p_2\cdots p_r.
\]
A més, aquesta descomposició és única, llevat de la possible reordenació dels factors.
\end{theorem}

\begin{remark}
Fixem-nos que això no passa en altres anells commutatius. Per exemple, a $R=\Z[\sqrt{-5}]$ l'element $6\in R$ es pot escriure com $6=2\cdot 3=(1+\sqrt{-5})(1-\sqrt{-5})$, i cadascun dels quatre elements $2$, $3$, $1+\sqrt{-5}$ i $1-\sqrt{-5}$ té exactament dos divisors llevat de les unitats $\pm 1$ (serien ``primers'' amb la definició que hem donat, però s'anomenen \emph{irreductibles}). Per tant, en aquest anell no es compleix l'anàleg del teorema fonamental de l'aritmètica.
\end{remark}

El resultat clau que ens caldrà per demostrar aquest teorema és el següent.
\begin{theorem}[Euclides]
\label{thm:Euclides-pab}
 Sigui $p$ és un primer i $a,b\in\Z$. Aleshores
 \[
 p\mid ab \implies p\mid a \text{ o } p \mid b.
\]
\end{theorem}
\begin{proof}
 Si $p\mid a$ ja estem. Si no, aleshores $\gcd(p,a)=1$. Per tant, $\gcd(pb,ab)=b$. Ara observem que $p\mid pb$ i $p\mid ab$, i per tant $p\mid \gcd(pb,ab)=b$.
\end{proof}

Ara ja podem demostrar el teorema fonamental de l'aritmètica.
\begin{proof}[Demostració del Teorema~\ref{thm:tfa}]
 Primer veiem l'existència, per inducció en $n\geq 1$. Si $n=1$ ja estem (producte buit). Pel cas general, si $n$ és primer ja estem (producte d'un sol terme), i si no, aleshores es pot escriure $n=ab$ amb $a,b < n$. Per hipòtesi d'inducció, tant $a$ com $b$ són producte de primers, i per tant $n$ també ho és.
 
 Per veure la unicitat, suposem que tenim dues factoritzacions
 \[
 n = p_1\cdots p_r = q_1\cdots q_s,
 \]
 amb els $p_i$'s i $q_j$'s primers. Observem que $p_1$ divideix $q_1\cdot(q_2\cdots q_s)$. Aleshores,  o bé $p_1=q_1$ o bé $p_1\mid q_2\cdots q_s$. Continuant, podem veure que $p_1 = q_j$ per algun $j$. Per tant, podem cancel·lar $p_1$ de la primera expressió i $q_j$ de la segona. Obtenim que
 \[
 n/p_1 = p_2\cdots p_r = q_1\cdots q_{j-1} q_{j+1}\cdots q_s.
 \]
 Per inducció sobre $n$, aquestes dues factoritzacions  de $n/p_1$ són iguals, i per tant les dues factoritzacions de $n$ també ho són.
\end{proof}

Tal i com hem vist a la primera part de la demostració, escriure una factorització en primers és fàcil si sabem trobar un primer que divideixi $n$ (o un factor no trivial). Aquest problema no és gens fàcil de fer quan $n$ és gran, i més endavant veurem la importància que aquest fet té per la criptografia.


El teorema fonamental de l'aritmètica ens porta a pensar que hi hauria d'haver molts primers, si amb ells s'han de poder construir tots els naturals. En efecte, tenim el següent resultat famós.

\begin{theorem}[Euclides]
 Hi ha infinits primers.
\end{theorem}
\begin{proof}
 Donats primers $p_1$, $p_2$, \ldots, $p_n$, construirem un primer $p_{n+1}$ diferent de tots els anteriors: considerem
 \[
 N= p_1 p_2\cdots p_n +1,
 \]
 i sigui $q$ un primer que divideixi a $N$. Aleshores $q\mid N$ i, si $q$ fos un dels $p_i$, aleshores $q$ també dividiria a $p_1p_2\cdots p_n = N-1$. Però això voldria dir que $q$ dividiria a $N-(N-1)=1$, que no pot ser. Per tant, $q$ és un primer que no apareix a la llista, i podem definir $p_{n+1} = q$. Com que aquest procés es pot repetir indefinidament, hi ha  infinits primers.
\end{proof}

També ens podem preguntar si podem trobar molts primers entre els termes d'una successió aritmètica donada. Concretament, si $a$ i $r$ són dos enters positius, podem considerar els enters de la forma $a + rx$, amb $x\geq 0$. Òbviament, si $g=\gcd(a,r)>1$, tindrem $g\mid a+rx$ i per tant com a molt hi haurà un primer en el conjunt $\{a+rx ~|~ x\geq 0\}$. En canvi, tenim el següent resultat, del qual no tindrem temps de fer la demostració.
\begin{theorem}[Dirichlet]
Siguin $a,r$ dos enters coprimers. Aleshores hi ha infinits primers de la forma $a+rx$, amb $x\in\Z$.
\end{theorem}


Si ens interessa enumerar els primers, podem fer servir l'anomenat \emph{garbell d'Eratòstenes}, que ens dona tots els primers menors que un enter donat $n$. Es tracta d'anar traient de la llista tots els múltiples de $p$, on $p$ és el primer element de la llista (que forçosament haurà de ser primer). Només cal mirar fins a $\sqrt{n}$, ja que si un enter $m$ no és primer, aleshores necessàriament ha de tenir un factor primer menor que $\sqrt{m}$.

\begin{algo}
\caption{Retorna una llista dels primers menors que $n$}
\begin{python}
def garbell(n):
    if n <= 2:
        return []
    elif n == 3:
        return [2]
    else:
        P = [2] # El primer més petit és el 2
        X = range(3,n,2) # Inicialitzem amb els senars < n.
        p = X[0]
        while p * p <= n:
            P.append(p)
            X = [x for x in X if x % p != 0]
            p = X[0]
        P += X
        return P
\end{python}
\end{algo}

 \subsection{Els enters mòdul \texorpdfstring{$n$}{n}}
 Donat un enter positiu $n$, considerarem el morfisme d'anells
 \[
 \red\colon \Z\to\Z/n\Z,\quad a\mapsto a\bmod{n}.
 \]
 Direm que $a\equiv b\pmod{n}$ si $\red(a)=\red(b)$ (com a elements de $\Z/n\Z$). És a dir, si $n\mid a-b$.
 
 \begin{proposition}[Cancel·lativitat]
 \label{prop:cancellativitat}
 Si $\gcd(c,n)=1$ i $ac\equiv bc\pmod{n}$, llavors $a\equiv b\pmod{n}$.
 \end{proposition}
\begin{proof}
 Farem servir el teorema fonamental de l'aritmètica: suposem que $n$ divideix $ac-bc = (a-b)c$ i $\gcd(c,n)=1$. Aleshores, si una potència d'un primer $p$ divideix exactament a $n$ (que escriurem $p^k\parallel n$), necessàriament $p^k\parallel (a-b)$ (ja que $p\nmid c$). Per tant, $n\mid (a-b)$, que és equivalent a $a\equiv b\pmod n$.
\end{proof}
 \subsubsection{Inversos mòdul \texorpdfstring{$n$}{n}}
 Considerem el grup d'unitats $(\Z/n\Z)^\times$ de l'anell $\Z/n\Z$. Ens interessa saber quins elements de $\Z/n\Z$ són unitats.
 
% \begin{definition}
% Un conjunt $R\subset \Z$ es diu que és un \emph{sistema complet de residus (SCR)} si la restricció de $\red$ a $R$ és una bijecció de conjunts.
% \end{definition}
% \begin{lemma}
% Si $R$ és un SCR i $\gcd(a,n)=1$, aleshores $aR=\{ax ~|~ x\in R\}$ també és un SCR.
% \end{lemma}
% \begin{proof}
%  Fixem-nos que $\# aR = \# R$, i que per la cancel·lativitat
% \[
% ax \equiv a x'\pmod n\implies x\equiv x'\pmod n,
% \]
% i com que $R$ és un SCR tenim que $x=x'$. Per tant els elements de $aR$ tenen tots ells diferents reduccions mòdul $n$.
% \end{proof}
 
 \begin{proposition}
 Si $\gcd(a,n)=1$, aleshores l'aplicació
 \[
 m_a\colon \Z/n\Z\to\Z/n\Z,\quad x\mapsto ax
 \]
 és una bijecció.
 \end{proposition}
 \begin{proof}
Com que $m_a$ és un morfisme de grups (amb la suma), podem parlar del nucli $\ker m_a$. Fixem-nos que, com que $\gcd(a,m)=1$, la Proposició~\ref{prop:cancellativitat} ens garanteix
\[
ax\equiv 0\pmod{m}\implies x\equiv 0\pmod{m},
\]
i per tant $\ker m_a=\{0\}$, i $m_a$ és injectiva. Com que els conjunts de sortida i d'arribada són finits i iguals, necessàriament $m_a$ és exhaustiva.
 \end{proof}
 
 \begin{corollary}[Unitats de $\Z/n\Z$]
 El grup d'unitats de $\Z/n\Z$ és
 \[
 (\Z/n\Z)^\times = \{ a\in\Z/n\Z ~|~ \gcd(a,n)=1\}.
 \]
 \end{corollary}
 \begin{proof}
  Si $\gcd(a,m)=1$, aleshores la proposició anterior (de fet, l'exhaustivitat d'$m_a$) ens garanteix l'existència d'un element $x$ tal que $ax\equiv 1\pmod{m}$, és a dir que $a$ és invertible a $\Z/m\Z$.
  
  Recíprocament, si $x\in\Z$ satisfà $ax\equiv 1\pmod{m}$, aleshores existeix $y\in\Z$ tal que
  \[
  ax+my = 1.
  \]
  Suposem que $d\mid a$ i $d\mid m$. Per l'equació anterior, $d\mid ax+my=1$, i per tant $d=1$. Concloem que $\gcd(a,m)=1$.
 \end{proof}
 
 \begin{remark}
 Fixem-nos que si $a\in\Z/n\Z$ té sentit parlar de $\gcd(a,n)$, pensant en $\gcd(\hat a,n)$ on $\hat a$ és un enter qualsevol tal que $\red(\hat a)=a$. Si prenem un altre aixecament $\hat a+ tn$, aleshores $\gcd(\hat a+tn,n)=\gcd(\hat a,n)$, i per tant no depèn de quin hem triat.
 \end{remark}
 Donarem un nom al cardinal d'aquest grup finit.
 \begin{definition}
 La funció \emph{$\varphi$ d'Euler} assigna a un enter positiu $n$ el valor
 \[
 \varphi(n)=\#(\Z/n\Z)^\times=\#\{1\leq a\leq n~|~ \gcd(a,n)=1\}.
 \]
 \end{definition}
 Per exemple, $\varphi(p) = p-1$ si $p$ és primer i, de fet, és fàcil de veure que, per tot $k\geq 1$,
 \[
 \varphi(p^k) = p^k - p^{k-1} = p^{k-1}(p-1).
 \]
 Més endavant veurem com determinar $\varphi(n)$ en general, si coneixem la factorització d'$n$ en producte de primers.
 
 El que hem desenvolupat fins aquí ens permet resoldre totes les equacions lineals mòdul $n$.
 
 \begin{proposition}
 \label{prop:eqslineals}
 L'equació $ax\equiv b\pmod n$ té  solució si i només si $\gcd(a,n)\mid b$.
 \end{proposition}
 \begin{proof}
  Sigui $g=\gcd(a,n)$. Si $x$ és una solució de $ax\equiv b\pmod n$, aleshores $n\mid ax-b$. Com que $g\mid a$ i $g\mid n$, aleshores $g\mid b$.
  
  Recíprocament, suposem que $g\mid b$. Aleshores $g\mid a$, $g\mid b$, i $g\mid n$. Per tant, $n\mid (ax-b)$ si i només si
  \[
  \frac n g\mid \left(\frac a g x - \frac b g\right).
  \]
  Però ara $\gcd(a/g,n/g)=1$ i, per tant, es té una solució de $a/g x \equiv b/g \pmod{n/g}$.
 \end{proof}

 \subsubsection{El petit teorema de Fermat i el teorema d'Euler}
 Recordem un teorema bàsic de la teoria de grups, conegut com el teorema de Lagrange: si $G$ és un grup finit aleshores l'ordre de qualsevol subgrup $H\subseteq G$ divideix l'ordre de $G$. Això ens servirà per demostrar dos teoremes atribuits a Euler i Fermat:
 \begin{theorem}[Euler]
 \label{thm:Euler}
 Sigui $\gcd(a,n)=1$. Aleshores
 \[
 a^{\varphi(n)}\equiv 1\pmod{n}.
 \]
 \end{theorem}
 \begin{proof}
  Considerem $G=(\Z/n\Z)^\times$, i $H=\langle a\rangle\subseteq G$. Aleshores el teorema de Lagrange ens diu que $\#H = \operatorname{ord}(a) \mid \#G = \varphi(n)$. Per tant, $a^{\varphi(n)}$ és la identitat a $G$, com volíem veure.
 \end{proof}
 
 \begin{corollary}[Petit teorema de Fermat]
 Si $p$ és un primer i $p\nmid a$, aleshores
 \[
 a^{p-1} \equiv 1\pmod{p}.
 \]
 \end{corollary}
 
 \begin{remark}
 Observem que el recíproc no és cert. Per exemple, $2^{340}\equiv 1 \pmod{341}$ i $341=11\cdot 31$. En aquest cas, però, es compleix $3^{340}\equiv 56 \pmod{341}$. Ens podem plantejar, doncs si és cert que si $n$ és compost aleshores podem trobar un enter $a$ coprimer amb $n$ tal que $a^{n-1}\not\equiv 1 \pmod{n}$. Això tampoc és cert, i als nombres que incompleixen aquest principi (per exemple $561=3\cdot 11 \cdot 17$) se'ls anomena \emph{nombres de Carmichael}. Més endavant n'estudiarem algunes de les seves propietats.
 \end{remark}
 \subsubsection{El teorema de Wilson}
 
 El següent resultat ens dona una caracterització dels primers en termes de congruències.
\begin{proposition}[Teorema de Wilson]
 Un enter $n>1$ és primer si i només si
 \[
 (n-1)!\equiv -1\pmod{n}.
 \]
\end{proposition}
\begin{proof}
 Suposem primer que la congruència és certa però que $n$ no és primer. Prenem llavors un factor primer $\ell\mid n$ de $n$. Tenim, per una banda, que $\ell\mid (n-1)!$, i també que $\ell\mid n\mid (n-1)!+1$. Però aleshores $\ell\mid 1$, que és una contradicció.
 
 D'altra banda, si $n>2$ és primer (per $n=2$ ho podem verificar directament), aleshores fixem-nos que els factors que apareixen en el producte
 \[
 (n-1)! = \prod_{a=1}^{n-1} a
 \]
 són representants de tots els elements de $(\Z/n\Z)^\times$. En particular, per cada $a$ que apareix en el producte també apareix $b\equiv a^{-1} \pmod n$, que es cancel·larà. L'única manera que aquests dos termes no es cancel·lin és si $b=a$, és a dir si $a^2\equiv 1\pmod n$, i això només passa per $a=1$ i $a=n-1$. Per tant, tenim $(n-1)!\equiv n-1\equiv -1\pmod n$, com volíem demostrar.
\end{proof}
Fixem-nos que no és un mètode pràctic per decidir si $n$ és primer, ja que per calcular el factorial de $n-1$ calen prop de $n$ operacions. Més endavant veurem altres mètodes que ens permetran demostrar que un nombre és compost, o bé donar-nos molta seguretat sobre el fet que és primer (si ho és).

  \subsubsection{El teorema xinès dels residus}
  
  \begin{theorem}
  \label{thm:crt}
  Si $m_1$, \ldots, $m_k$ són enters coprimers dos a dos, aleshores el morfisme d'anells
  \[
  \Z/(m_1\cdots m_k)\Z\to\Z/m_1\Z \times\cdots\times \Z/m_k\Z,\quad a\mapsto (a\bmod m_1,\ldots, a\bmod m_k)
  \]
  és un isomorfisme.
  \end{theorem}
  
  La demostració d'aquest teorema es redueix fàcilment, com veurem, al cas $k=2$. En aquest cas, podem veure que el teorema es diu el següent:
  \begin{proposition}
   Siguin $m,n\in\Z$ enters amb $\gcd(m,n)=1$. Aleshores, donats $a,b\in \Z$, el sistema d'equacions
   \begin{align*}
   x&\equiv a\pmod{m}\\
   x&\equiv b\pmod{n}
   \end{align*}
   té solució, que és única mòdul $mn$.
  \end{proposition}
  \begin{proof}
  Busquem $x$ de la forma
  \[
  x = a+tm,
  \]
  per algun $t$ tal que $a+tm\equiv b\pmod{n}$. Aquesta equació té solució perquè $\gcd(m,n)=1$, tal i com hem vist a la Proposició~\ref{prop:eqslineals}.
  
  Per veure la unicitat, considerem dues solucions $x$ i $y$. Aleshores $z=x-y$ és divisible per $n$ i $m$. Com que $\gcd(m,n)=1$, tenim $nm\mid z$, i per tant $z\equiv 0\pmod{mn}$, d'on tenim que $x\equiv y\pmod{mn}$.
  \end{proof}

  \begin{proof}[Demostració (del Teorema~\ref{thm:crt})]
   Farem inducció en $k\geq 1$, on el cas $k=1$ és trivial. Considerarem $k\geq 2$. Per veure l'exhaustivitat cal trobar, donats $a_1,\ldots,a_k$, un enter $x\in\Z$ tal que
   \begin{align*}
   x&\equiv a_1\pmod{m_1}\\
   x&\equiv a_2\pmod{m_2}\\
   \phantom{x}&\phantom{\equiv}\vdots\\
   x&\equiv a_k\pmod{m_k}.
   \end{align*}
   Aplicant la proposició anterior,
   el conjunt de solucions de les dues primeres equacions és el mateix que el conjunt de solucions de
   \[
   x\equiv a_{12}\pmod{m_1m_2},
   \]
   on $a_{12}$ és la solució proporcionada per la Proposició. Per tant, ens reduïm al sistema
    \begin{align*}
   x&\equiv a_{12}\pmod{m_1m_2}\\
   \phantom{x}&\phantom{\equiv}\vdots\\
   x&\equiv a_k\pmod{m_k}.
   \end{align*}
  que té una solució única mòdul $m_1m_2\cdots m_k$, per hipòtesi d'inducció.
  \end{proof}
 
 Tenim una versió del teorema xinès dels residus pels grups d'unitats.
 \begin{lemma}
 Si $m_1$, \ldots, $m_k$ són enters coprimers entre si, aleshores el morfisme de grups
  \[
  (\Z/(m_1\cdots m_k)\Z)^\times\to(\Z/m_1\Z)^{\times} \times\cdots\times (\Z/m_k\Z)^{\times},\quad a\mapsto (a\bmod m_1,\ldots, a\bmod m_k)
  \]
  és un isomorfisme.
 \end{lemma}
 \begin{proof}
  Si $\gcd(a,m_1\cdots m_k) = 1$, aleshores $\gcd(a,m_i)=1$ per a tot $i=1,\ldots,k$. Per tant, l'aplicació està ben definida.
  
  La injectivitat és automàtica, pel fet que es tracta de la restricció del morfisme d'anells del teorema xinès dels residus.
  
  Per veure l'exhaustivitat, observem que el teorema xinès dels residus ens garanteix, donats $a_i \in \Z/m_i\Z$, un element $a\in\Z/(m_1\cdots m_k)\Z$ que tal que $a\pmod m_i = a_i$. Ara bé, si sabem que $\gcd(a_i,m_i)=1$ per a tot $i$, aleshores $\gcd(a,m_i)=1$ per a tot $i$, i d'aquí obtenim (pel Teorema~\ref{thm:Euclides-pab}) que $\gcd(a,m)=1$.
 \end{proof}
 
En teoria de nombres, una funció $f$ s'anomena \emph{multiplicativa} si $f(mn)=f(m)f(n)$ sempre i quan $\gcd(m,n)=1$. Si $f(mn)=f(m)f(n)$ per a tot $m,n$ aleshores s'anomena \emph{completament multiplicativa}.

 \begin{corollary}
 La funció $\varphi$ d'Euler és multiplicativa: si $\gcd(m,n)=1$, aleshores
 \[
 \varphi(mn)=\varphi(m)\varphi(n).
 \]
 \end{corollary}
 \begin{proof}
  Només cal prendre cardinalitats en el lema anterior.
 \end{proof}
Com a conseqüència de la multiplicativitat de $\varphi$, podem donar una fórmula per $\varphi(n)$ en termes de la factorització de $n$.
\begin{proposition}
 Sigui $n\geq 1$ un enter que factoritza com $n = p_1^{e_1} p_2^{e_2}\cdots p_k^{e_k}$. Aleshores
\[
\varphi(n)= n \prod_{i=1}^k \left(1-\frac{1}{p_i}\right) = \prod_{i=1}^k p_i^{e_i-1}(p_i-1).
\]
\end{proposition}

\begin{remark}
\label{rmk:factoritzar-i-phi}
En general, és difícil calcular $\varphi(n)$ eficientment sense conèixer una factorització de $n$. Per exemple, si $n=pq$ és el producte de dos primers, aleshores la informació que ens dona saber $\varphi(n)$ ens permet calcular la factorització de $n$ de manera molt ràpida: considerem el polinomi $X^2-2sX + n$, on $s = \frac{n+1-\varphi(n)}{2}$. Aquest polinomi té $p$ i $q$ com a arrels, que podem trobar calculant:
\[
p,q = s \pm \sqrt{s^2 - n}.
\]
\end{remark}
La funció $\varphi$ també satisfà una propietat que ens serà útil més endavant.

\begin{proposition}
 Per a tot $n\geq 1$ es té:
 \[
 \sum_{\substack{d\geq 1\\d \mid n}}\varphi(d) = n
 \]
\end{proposition}
\begin{proof}[Demostració 1 (fent servir teoria de grups)]
Observem que per a tot $d$, $\varphi(d)$ és el nombre de generadors d'un grup cíclic d'ordre $d$. Considerem doncs un grup cíclic $G$ d'ordre $n$. Diem que dos elements $x,y\in G$ estan relacionats si $\langle x\rangle = \langle y \rangle$. El nombre d'elements a la classe de $x$ és $\varphi(d)$ si $x$ té ordre $d$, i $G$ té un únic subgrup d'ordre $d$ per cada divisor $d\mid n$, d'on obtenim la fórmula.
\end{proof}
\begin{proof}[Demostració 2 (elemental)]
 Anomenem $f(n)$ al terme de l'esquerra, i volem veure que $f(n)=n$. Primer veurem que $f$ és multiplicativa: considerem enters coprimers $m$ i $n$. Donat un enter $k$, denotem per $\Delta(k)$ el conjunt dels seus divisors positius. Aleshores es té una bijecció $\Delta(m)\times\Delta(n)\to \Delta(mn)$, donada per $(d_1,d_2)\mapsto d_1d_2$ (comproveu-ho). Per tant:
 \begin{align*}
 f(mn)&=\sum_{d\in\Delta(mn)}\varphi(d) = \sum_{d_1\in\Delta(m)}\sum_{d_2\in\Delta(n)}\varphi(d_1d_2)\\
 &=\sum_{d_1\in\Delta(m)}\sum_{d_2\in\Delta(n)}\varphi(d_1)\varphi(d_2)\\
 &=\sum_{d_1\in\Delta(m)}\varphi(d_1)\sum_{d_2\in\Delta(n)}\varphi(d_2) = f(m)f(n).
 \end{align*}
 Per tant, només cal comprovar que $f(p^k) = p^k$ per a tot primer $p$ i tot $k\geq 1$. Els divisors de $p^k$ són de la forma $p^r$ amb $0\leq r \leq k$, i per tant:
\[
f(p^k) = \sum_{r=0}^k \varphi(p^r) = 1+\sum_{r=1}^k (p-1)p^{r-1} = p^k.
\]
\end{proof}
 \subsection{Mètodes efectius per inversos i exponenciació}
 El primer que veurem és com es pot trobar de manera efectiva l'invers d'un element $a$ mòdul $n$, és a dir, com resoldre l'equació $ax\equiv 1\pmod{n}$, suposant que $\gcd(a,n)=1$. L'eina clau ens la dona el que es coneix com la identitat de Bézout.
 
 \begin{proposition}[Identitat de Bézout]
 \label{prop:xgcd}
  Siguin $a,b\in\Z$ i $g=\gcd(a,b)$. Aleshores existeixen $x,y\in\Z$ tals que
  \begin{equation}
  \label{prop:bezout}
  g=ax+by.
  \end{equation}
 \end{proposition}
 
 Com que la demostració es pot fer constructiva, començarem amb un exemple, que podrem convertir en un algoritme que ens proporcioni la demostració.
 
 \begin{example}
 Prenem $a=120$, $b = 53$. Ja veiem que $\gcd(a,b)=1$, però el que farem serà aplicar l'algoritme d'Euclides i aprofitar tota la informació que ens dona:
 \begin{align*}
120 &= \ul 2\cdot 53 + 14\\
53 & = \ul 3\cdot 14 + 11\\
14 &= \ul 1 \cdot 11 + 3\\
11 &= \ul 3\cdot 3 + 2\\
3 &= \ul 1\cdot 2 + 1
 \end{align*}
 Ara aprofitem les equacions anteriors, per escriure:
 \begin{align*}
 14 &= 120 - 2\cdot 53\\
 11 &= 53 - 3\cdot 14 = 53 - 3\cdot(120 - 2\cdot 53) = -3\cdot 120 + 7\cdot 53\\
 3 &= 14 - 1\cdot 11 = (120 - 2\cdot 53) -1\cdot(-3\cdot 120 +7\cdot 53) = 4\cdot 120 -9\cdot 53\\
 2 &= 11 - 3\cdot 3 = (-3\cdot 120 + 7\cdot 53) - 3\cdot (4\cdot 120 - 9\cdot 53) = -15\cdot 120 + 34\cdot 53\\
 1 &= 3 - 1\cdot 2 = (4\cdot 120 - 9\cdot 53) - 1\cdot(-15\cdot 120 + 34\cdot 53) = 19\cdot 120 - 43\cdot 53.
 \end{align*}
 Fixem-nos que podem rescriure les igualtats anteriors fent servir ``coordenades'' respecte la parella $(120, 53)$:
 \begin{align*}
     14 &= (1,0) - \ul{2}\cdot (0,1) = (1,-2)\\
     11 &= (0,1) - \ul{3}\cdot (1,-2) = (-3,7)\\
     3 &=  (1,-2) - \ul{1}\cdot(-3,7) = (4,-9)\\
     2 &= (-3,7) - \ul{3}\cdot(4,-9) = (-15,34)\\
     1 &= (4,-9) - \ul{1}\cdot(-15,34) = (19,-43)
 \end{align*}
 Observem que els nombres subratllats són justament els quocients que hem anant obtenint en les divisions successives.
 \end{example}
 L'exemple anterior ens dona la idea de l'algoritme conegut com ``algoritme d'Euclides extès'':
 \begin{algo}
 \caption{Donats $a,b>0$, retorna enters $g$, $x$ i $y$ satisfent $g=\gcd(a,b)$ i $ax+by=g$}
 \begin{python}
 def xgcd(a,b):
    x, y, r, s = 1, 0, 0, 1
    while b:
        q, c = a.quo_rem(b)
        a, b, r, s, x, y = b, c, x - q * r, y - q * s, r, s
    return a, x, y
 \end{python}
 \end{algo}
 
 \begin{proof}[Demostració (de la Proposició~\ref{prop:xgcd})]
    Demostrarem que l'algoritme és correcte. Denotem els valors inicials per $a_0, b_0, r_0, s_0, x_0, y_0$ i els valors després de $n$ iteracions per $a_n, b_n, r_n, s_n, x_n, y_n$. Podem suposar que $a_0, b_0\geq 0$, i veurem per inducció que a cada iteració es té que
    \begin{align*}
    a_n &= a x_n + b y_n\\
    b_n &= a r_n + b s_n\\
    \gcd(a_n,b_n)&=\gcd(a,b)
    \end{align*}
    Fixem-nos que el cas $n=0$ és trivial. Ara bé:
    \begin{enumerate}
        \item $a_{n+1} = b_n$, $x_{n+1} = r_n$, $y_{n+1}=s_n$, i per tant $(1)$ es redueix a observar que $b_n = a r_n + b s_n$, per hipòtesi d'inducció.
        \item $b_{n+1} = c = a_n - qb_n$, i $r_{n+1} = x_n-qr_n$, $s_{n+1} = y_n-qs_n$. Per tant, $(2)$ es redueix a observar que
        \begin{align*}
        ar_{n+1}+bs_{n+1}&=a(x_n-qr_n) + b(y_n-qs_n)\\
        &=ax_n+by_n -q(ar_n+bs_n)\\
        &=a_n-qb_n =b_{n+1}.
        \end{align*}
        \item Per hipòtesi d'inducció, tenim $\gcd(a_n,b_n)=\gcd(a,b)$. Aleshores:
        \begin{align*}
        \gcd(a_{n+1},b_{n+1})&=\gcd(b_n,a_n-qb_n)\\
        &=\gcd(b_n,a_n)=\gcd(a,b).
        \end{align*}
    \end{enumerate}
Quan l'algoritme acaba, $b_n=0$ i per tant $\gcd(a,b)=\gcd(a_n,0)=a_n$. A més, $a_n = ax_n+by_n$, i per tant $x=x_n$ i $y=y_n$ satisfan la identitat que busquem.
 \end{proof}

  \begin{remark}
  Fixem-nos que la solució del teorema xinès dels residus es troba invertint $m$ mòdul $n$, i per tant es basa en última instància en l'algoritme d'Euclides extès. Més concretament, com que $\gcd(m,n)=1$, podem trobar enters $x,y$ tals que
  \[
  xm+yn=1.
  \]
  Aleshores podem definir $z=a + (b-a)xm=ayn + bxm$. Fixem-nos que:
  \[
  ayn + bxm \equiv ayn \equiv a(1-xm)\equiv a\pmod m,
  \]
  i
  \[
  ayn + bxm \equiv bxm \equiv b(1-yn)\equiv b\pmod n.
  \]
  \end{remark}
 Fent servir la identitat de Bézout, és molt fàcil donar un algoritme per resoldre $ax = b\pmod m$:
 
 \begin{algo}
 \caption{Donats $a$, $b$ i $m$ retorna $x$ satisfent $ax\equiv b\pmod m$}
 \begin{python}
 def resol_equacio_lineal(a,b,m):
    g, x, y = xgcd(a, m) # g = a * x + m * y
    q, r = b.quo_rem(g)
    if r != 0:
        raise ValueError("L'equació no té solució")
    else:
        return q * x
 \end{python}
  \end{algo}
 
 En particular, podem calcular inversos a $\Z/m\Z$:
 
 \begin{algo}
 \caption{Donats $a$ i $m$ coprimers, retorna $a^*$ satisfent $aa^*\equiv 1\pmod m$}
 \begin{python}
 def invers_mod(a,m):
     return resol_equacio_lineal(a,1,m)
 \end{python}
 \end{algo}
 
  El segon objectiu que ens proposem en aquesta secció és el de, donats enters $a$, $r$ i $m$, calcular la quantitat
  \[
  a^r\pmod{m}.
  \]
  Com que ja sabem calcular inversos, suposarem que $r>0$. Aleshores, podem suposar d'entrada que: $m\geq 2$ i que $0\leq a\leq m$.
 
  La manera naïf de calcular $a^r\pmod{m}$ consistiria en calcular primer $a^r$ i després reduir mòdul $m$. Si $r$ és gran, però, això ens faria treballar amb nombres molt grans (nombres amb $r$ vegades el nombre de dígits d'$a$), mentre que el resultat és petit (busquem un nombre menor que $m$). Per tant, a cada operació ens interessa reduir el resultat parcial mòdul $m$.
  
 L'altre problema que tenim és que, si $r$ és gran, aleshores hauriem d'evitar fer $r-1$ multiplicacions (que és com probablement hem apres a calcular $a^r$). En l'exemple següent veiem com podem fer-ho més ràpidament.
 
 \begin{example}
  Suposem que volem calcular $a^{25}$. Observem que $25=16+8+1 = 2^4+2^3+1$. Per tant,
  \[
  a^{25} = a^{2^4+2^3+1}=a^{2^4}\cdot a^{2^3}\cdot a = (((a^2)^2)^2)^2\cdot ((a^2)^2)^2\cdot a.
  \]
  Aleshores, podem obtenir el resultat calculant primer $a^2$, després $a^4 = (a^2)^2$, després $a^8=(a^4)^2$, després $a^{16}=(a^8)^2$ i, finalment $a^{25}=a^{16}\cdot a^8\cdot a$ s'obté fent dos productes de les quantitats prèvies. En total, hem elevat al quadrat $4$ vegades i hem fet $2$ multiplicacions al final: aquestes $6$ multiplicacions són bastant menys que les $24$ que haurien calgut per obtenir el resultat de manera naïf.
 \end{example}
 
 Donem un algoritme que calcula $a^r\pmod{m}$ amb $O(\log(r))$ multiplicacions a $\Z/m\Z$.
 
 \begin{algo}
 \caption{Calcula $a^r\pmod m$, versió inicial}
\begin{python}
def exponentiate(a,r,m):
    result = 1
    powers = a % m
    while r:
        if r % 2 == 1:
            result = (result * powers) % m
        powers = powers ** 2 % m
        r //= 2
    return result
\end{python}
  \end{algo}
  
 Podem estalviar espai llegint els bits al revés. Vegem primer un exemple
 \begin{example}
 Escrivim
 \[
 a^{25} = a^{16+8+1} = a^{16+8}\cdot a = (a^{4}\cdot a^2)^4\cdot a = (((a^2\cdot a)^2)^2)^2\cdot a.
 \]
 En aquest cas, amb una sola variable podem anar desant el resultat parcial.
 \end{example}
 
 Obtindrem la següent funció:
  \begin{algo}
 \caption{Calcula $a^r\pmod m$, versió millorada}
 \begin{python}
def exponentiate_reverse(a,r,m):
    result = 1
    for bit in reversed(r.bits()):
        result = result**2 % m
        if bit == 1:
            result = (result * a) % m
    return result
 \end{python}
 \end{algo}

\subsection{Aplicacions a la criptografia}
Els algoritmes que hem vist fins ara ens permeten descriure protocols clàssics en la criptografia. Un d'aquests (RSA) porta dècades utilitzant-se a la pràctica.

\subsubsection{Diffie--Hellman}
\label{sec:diffie-hellman}
L'intercanvi de claus de Diffie--Hellman  funciona de la manera següent. Les dues parts, Alice i Bob, fixen un grup cíclic $G$, de cardinal $N$. Alice i Bob fixen també un generador $g\in G$, que a l'igual que $G$ (i $N$) també serà públic. El protocol funciona de la manera següent:
\begin{enumerate}
    \item Alice escull un enter a l'atzar $1<a<N$, i envia la quantitat $A=g^a$ a Bob.
    \item Bob, per la seva banda, escull un enter a l'atzar $1<b<N$, i envia la quantitat $B=g^b$ a l'Alice.
    \item Alice i Bob calculen respectivament $S_a=B^a$ i $S_b=A^b$. Observem que els dos elements són iguals a $S=g^{ab}\in G$, que serà el secret compartit.
\end{enumerate}

Primer de tot, observem que els càlculs involucrats es poden fer de manera eficient gràcies a l'exponenciació modular, si tenim una manera eficient de calcular en $G$.

Un possible exemple de grup cíclic que funciona en aquest cas és $G = (\Z/p\Z)^\times$ i $N=p-1$ (on $p$ és un primer). En aquest cas, per trobar un generador $g$ de manera fàcil, el que es fa és triar un primer $p$ de la forma $p=2q+1$ amb $q$ primer, i així per veure que $g\neq \pm 1$ té ordre $p-1$ només cal comprovar (mitjançant exponenciació modular) que $g^q\equiv -1\pmod{p}$.

Fixem-nos que un observador Eve que tingui accés a tota la comunicació sap els valors de $g^a\pmod{p}$ i $g^b\pmod{p}$. El \emph{problema de Diffie--Hellman} consisteix a calcular $g^{ab}\pmod{p}$ donats $g^a\pmod{p}$ i $g^b\pmod{p}$. Al 2021, no coneixem\footnote{S'entén la comunitat acadèmica.} cap algoritme que resolgui aquest problema sense resoldre el \emph{problema del logaritme discret}: donats $A$ i $g$, trobar $a$ tal que $g^a\equiv A\pmod{p}$.

 \subsubsection{El xifrat ElGamal}
 \label{sec:elgamal}
 
 Una variant del protocol de Diffie--Hellman ens permet establir un sistema de xifrat de clau pública basat en el logaritme discret. 
 
 \begin{description}
 \item[Preparació:] Cada usuari tria un grup cíclic $G=\langle g \rangle$ de tamany $N$. Seguidament, tria un enter $2< s < N$ i calcula $K=g^s$. La clau pública de l'Alice serà la tupla $(G, N, g, K)$, i la clau secreta serà $s$.
 \item[Xifrat:] Suposem que l'Alice vol enviar un missatge $m\in G$ a en Bob, que té clau pública $(G, N, g, K_B)$. L'Alice tria un enter a l'atzar, $2< y < N$, i calcula $Y=g^y$ i també $Z=mK_B^y$. Aleshores envia a en Bob la tupla $(Y,Z)$.
 \item[Desxifrat:] Per recuperar el missatge, en Bob calcula $Y^{-s} Z$. Observem que
 \[
 Y^{-s}Z = g^{-sy}mg^{sy} = m.
 \]
 \end{description}
 Fixem-nos que la part de preparació s'assembla molt al que fa l'Alice en el protocol Diffie--Hellman, mentre que la part de xifrat s'assembla al que faria en Bob, amb la diferència que la part ``secreta'' $y$ va canviant a cada missatge. Aleshores la part ``pública'' $Y$ permet acordar un secret comú (que seria $Y^s=g^{sy}=K_B^y$), i que és justament el secret que s'ha fet servir per emmascarar el missatge $m$.
 
 Si un possible atacant que tingui accés a la comunicació vol recuperar el missatge $m$ a partir de $(Y,Z)$, haurà de trobar el secret a partir de $Y=g^y$ i de $K_B=g^s$, i això és precisament el problema de Diffie--Hellman, que estem assumint igual de difícil que el problema del logaritme discret.

\subsubsection{El xifrat RSA}
Com en el cas anterior, es tracta d'establir un protocol que permeti a qualsevol usuari d'escriure un missatge xifrat de manera que només el receptor pretés el pugui desxifrar.

\begin{description}

\item[Preparació:] Cada usuari escull dos primers grans $p$ i $q$, i calcula el producte $N=pq$ i $\varphi(N)=(p-1)(q-1)$. Aleshores tria un enter $1<e<\varphi(n)$, i fent servir l'algoritme que hem vist abans calcula el seu invers mòdul $\varphi(n)$: calcula $d$ amb $de\equiv 1\pmod{\varphi(n)}$. Així, cada usuari té una clau pública $(N,e)$ i una clau privada $(\varphi(N),d)$.

\item[Xifrat:] Suposem ara que l'Alice vol enviar un missatge a en Bob, que té clau pública $(N_B,e_B)$. Podem suposar que el missatge està codificat com un enter $1<m<N_B$ coprimer amb $N_B$. Aleshores l'Alice calcula $c=m^{e_B}\bmod{N_B}$, que serà el missatge xifrat.

\item[Desxifrat:] Per recuperar el missatge, en Bob calcula $c^{d_B}\bmod{N_B}$.
\end{description}
Vegem que
\[
c^{d_B}\equiv m^{e_Bd_B}\equiv m^{1+t\varphi(N_B)}\equiv m\pmod{N_B},
\]
i per tant en Bob pot desxifrar el missatge. Sense saber quant val $\varphi(N_B)$, no es pot trobar $d_B$ a partir de $e_B$ i, per tant, la seguretat del sistema rau en la dificultat de factoritzar $N_B$ (vegeu la Remarca~\ref{rmk:factoritzar-i-phi}). Més endavant veurem algoritmes per factoritzar enters, però els millors mètodes són sub-exponencials, fet que els fa inviables\footnote{Almenys al 2020! El febrer de 2020 un equip de 6 matemàtics va aconseguir factoritzar l'RSA-250, que té 250 dígits decimals.} per certs nombres de més de 260 decimals.  