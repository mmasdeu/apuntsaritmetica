\begin{enumerate}[leftmargin=*]
\item Sigui $p$ un nombre primer. Direm que $\alpha \in \F_p^\times$ és una
arrel primitiva si genera el grup $\F_p^\times$.

\begin{enumerate}
\item Sigui $p$ un nombre primer de la forma $p=2^{2^k}+1$
per algun $k$ (primers de Fermat). Demostreu que el conjunt d'arrels
primitives a $\F_p$ és igual al conjunt d'elements de $\F_p$ que no
són quadrats.

\item Sigui $q$ un primer tal que $p=4q+1$ també és primer. Demostreu que el conjunt
d'arrels primitives a $\F_p$ és igual al conjunt d'elements que no
són quadrats ni el seu quadrat és $-1$.

\item Trobeu totes les arrels primitives a $\F_{13}$,
$\F_{29}$, $\F_{149}$.

\item Demostreu que el nombre d'arrels primitives de $\F_p$ és
igual a $\varphi(\varphi(p))$.

\item Demostreu que si $p>3$ és un nombre primer, el producte
de totes les arrels primitives de $\F_p$ és igual a $1$.

\item Demostreu que si $p\ge 3$ és un nombre primer, la suma
de totes les arrels primitives de $\F_p$ és congruent a $\mu(p-1)
\pmod{p}$, on $\mu(n)$ és la funció de Möbius, que val $1$ si $n$ és
producte d'un nombre parell de primers diferents, $-1$ si $n$ és
producte d'un nombre senar de primers diferents, i $0$ si no (o
sigui $n$ és divisible per un quadrat $>1$).


\end{enumerate}


(Pista per e) i f): Demostreu que
$$\prod_{\alpha\mbox{\small{} arrel primitiva a } \F_p}
(x-\alpha)=\Phi_{p-1}(x),$$ on $\Phi_n(x)$ és el polinomi ciclotòmic enèssim.


\item Parametrització de corbes i varietats.

\begin{enumerate}

\item Calculeu una fórmula que determini les solucions
racionals de l'equació $ax^2+by^2=a+b$ en $(x,y)$, en funció del
paràmetres $a$ i $b$.

\item Calculeu una fórmula que determini les solucions
racionals de l'equació $x^2+y^2+z^2=1$ en $(x,y,z)$. Per fer-ho,
podeu utilitzar la projecció estereogràfica $\pi(x,y,z)=(s,t)$ a
partir del punt $(0,0,-1)$, i trobeu una descripció explícita del
punt $(x,y,z)$ en funció del punt $(s,t)$ del pla.

\item Feu el mateix que l'apartat (b) però amb
l'equació $x_1^2+x_2^2+\cdots+x_n^2=1$.

\item Comproveu que les solucions racionals de l'equació
$x^3+y^3+z^3=1$ venen donades donant valors racionals $(s,t)$ a les
fórmules $$ x(s,t) = \frac{3 t - \frac 13 (s^2 + s t + t^2)^2}{t
(s^2 + s t + t^2) - 3} $$

$$y(s,t) = \frac{3 s + 3 t + \frac 13 (s^2 + s t + t^2)^2}{t
(s^2 + s t + t^2) - 3}$$

$$ z(s,t) = \frac{-3 - (s^2 + s t + t^2) (s + t)}{t (s^2 + s t
+ t^2) - 3}.$$

\end{enumerate}

\item
\begin{enumerate}
\item Demostreu que si $n\ge 2$ i $k\ge 1$ són enters,
aleshores $(n-1)$ sempre divideix $(n^k-1)$, però que $(n-1)^2$
divideix $(n^k-1)$ si i només si $(n-1)$ divideix $k$.

\item Demostreu que $n$ divideix $(n-1)!+1$ si i només si $n$
és un nombre primer.

\item Demostreu que $(n-1)!+1=n^k$ per un cert nombre $k$ si i
només si $n=2,3$ o $5$.

\end{enumerate}

\item Un nombre enter $n$ no primer s'anomena un nombre de
Carmichael si per a tot nombre enter $a$ coprimer amb $n$ tenim que
$$a^{n-1} \equiv 1 \pmod{n}.$$

\begin{enumerate}
\item Demostreu que un nombre enter $n$ es de Carmichael si i
només si $n$ és compost i divideix $a^n-a$ per a tot enter $a$.

\item Demostreu que per a tot $n\ge 2$ enter, el nombre d'elements $a$ de $\Z/n\Z$ tals que $a^{n-1}=1$ és exactament igual a
$$\prod_{p \mid n, \ p \text{ primer}} \gcd(p-1,n-1)$$

\item Demostreu que un nombre compost $m$ es de Carmichael si
és lliure de quadrats i $(p-1)$ divideix $(m-1)$ per a tot primer
$p$ que divideix $m$.

\item Trobeu tots els nombres de Carmichael entre $2$ i $3000$.

\end{enumerate}

\item Considereu la funció $\varphi$ d'Euler (o sigui,
$\varphi(n)$ és igual al nombre d'elements invertibles a l'anell
$\Z/n \Z$).

\begin{enumerate}
\item Trobeu tots els nombres $n$ tals que $\varphi(n)=1$, i
tots els nombres $n$ tals que $\varphi(n)=2$.

\item Trobeu tots els nombres $n$ tals que $\varphi(n)=24$.

\item Trobeu l'enter positiu més petit $a$ tal que no hi ha
cap enter $n$ amb $\varphi(n)=a$, el més petit tal que té
exactament 2 solucions, exactament 3 solucions i exactament 4
solucions.

\item Demostreu que, si $\gcd(m,n)>1$, aleshores
$\varphi(nm)>\varphi(n)\varphi(m)$.

\item Demostreu que, si $m \mid n$, aleshores $\varphi(m)\mid
\varphi(n)$.

\item Demostreu que  $n \mid \varphi(a^n-1)$ per a tot $a>1$.
\end{enumerate}

\item Sigui $p$ un nombre primer senar.
\begin{enumerate}

\item Sigui $A$ el producte
de tots els residus quadràtics mòdul $p$ (els residus quadràtics
mòdul $p$ són els elements de $\F_p^\times$ que són quadrats). Demostreu
que 
\[
A\equiv (-1)^{(p+1)/2} \pmod{p}.
\]
\item Demostreu que si $p\equiv 1\pmod{4}$, aleshores la suma de tots el nombres $r$, $1 \le r \le p-1$ que
són residus quadràtics és igual a $p(p-1)/4$.

\item Sigui $a$ un nombre enter no divisible per $p$. Considerem el conjunt $$S_a=\{\overline{na}\in \F_p \ : \
n\in \{1,2,\dots,(p-1)/2\} \},$$ i el conjunt $S_a'$  format pels
elements $\bar r$ de $S_a$ tals que $r>p/2$, on $r$ és l'únic nombre
enter tal que $1\le r\le p-1$, i $r\equiv \bar r \pmod{p}$.
Demostreu que $$\prod_{r\in S_a} r=(-1)^{|S_a'|}\prod_{s\in S_1}s$$
i dedu\"{\i}u que $\left(\frac{a}{p}\right)=(-1)^{|S_a'|}$.
\end{enumerate}


\item Sigui $m$ un enter positiu
senar. Podem escriuré així $m=p_1\dots p_s$ on els $p_i$ són primers
senars, no necessàriament diferents. Es defineix el \textbf{símbol de Jacobi}
com:
$$
\displaystyle{\left (\frac{a}{m} \right )=\prod_{i=1}^{s}\left
(\frac{a}{p_i} \right )}.
$$

\begin{enumerate}
\item Demostreu que si $a$ és primer amb $m$, aleshores
l'aplicació $\mu_a\colon : \Z/m\Z\to \Z/m\Z$ definida com $x\mapsto ax$ és un automorfisme
(i.e. és bijectiva i morfisme de grups).

\item Denotem per  $\displaystyle{\left[\frac{a}{m} \right]}$ el signe de la permutació donada per
l'aplicació $\mu_a$ a $\Z/m\Z$. Demostreu que si $m=m_1m_2$,
aleshores
$$
\displaystyle{\left[ \frac{a}{m} \right] =\left [\frac{a}{m_1}
\right]\left [\frac{a}{m_2} \right]}.
$$

\item Demostreu que per a tot $m$ senar i $a$ primer amb $m$
tenim que
$$\displaystyle{\left[\frac{a}{m} \right]}=\displaystyle{\left(\frac{a}{m}
\right)}.$$

\item Siguin $m$ i $n$ dos enters senars positius i primer
entre si. Si escollim com a representant de cada element de $\Z/m\Z$
un enter $0\le a <m$, i també representants $0\le b<n$ dels
elements de $\Z/n\Z$, demostreu que tenim tres bijeccions naturals
$L,M,N\colon \Z/m\Z\times \Z/n\Z\to \Z/(mn)\Z$ donades per $L$ la
bijecció que ens determina el teorema xin\`{e}s, $N(a,b)=an+b$ i
$M(a,b)=bm+a$.

\item Demostreu que per a tot $b\in \Z/m\Z$, el signe de la permutació
$\sigma_b\colon\Z/m\Z\to \Z/m\Z$ que envia $\sigma_b(a)=a+b$ és
sempre 1.  Comproveu que la composició de $L^{-1}N$ és igual
a l'aplicació $(\sigma_b\circ \mu_n)\times i \colon\Z/m\Z\times
\Z/n\Z\to \Z/m\Z\times \Z/n\Z$ donada $((\sigma_b\circ \mu_n)\times
i)(a,b)=(na+b,b)$. Idem amb $L^{-1}M$ i $i\times (\sigma_a\circ
\mu_m)$.

\item Demostreu que el signe de la permutació de $\Z/m\Z\times \Z/n\Z$ donada per la
bijecció $N^{-1}M$ és $(-1)^{(m-1)(n-1)/4}$. Per fer-ho podeu
fer servir el fet que el signe d'una permutació $\tau$ d'un conjunt $X$
totalment ordenat és igual a $-1$ elevat al nombre d'inversions de
$\tau$: parelles $(a,b)\in X^2$ tals que $a<b$ i $\tau(b)<\tau(a)$,
i demostrar que si posem l'ordre lexicogràfic a $\Z/m\Z\times
\Z/n\Z$, el nombre d'inversions és $ \binom{m}{2}\binom{n}{2}$.

\item Dedu\"{\i}u de $(L^{-1}N)(N^{-1}M)=L^{-1}M$ la Llei de reciprocitat quadràtica per al símbol de
Jacobi:  Si $(m,n)=1$, aleshores
$$
\left(\frac{m}{n} \right) \left(\frac{n}{m}
\right)=(-1)^{\frac{m-1}{2}\frac{n-1}{2}}.
$$

\end{enumerate}

\item Direm que un nombre és $k$-compost si es divisible per
com a mínim $k$-nombres primers diferents. Per exemple, 24 es
2-compost però no és 3-compost, i $210=2\cdot 3 \cdot 5 \cdot 7$ es
$k$-compost per a $k=2,3,4$.

\begin{enumerate}
\item Trobeu 3 nombres consecutius cadascun d'ells 3-compost.

\item Demostreu que per a tot $k\ge 2$ i per a tot $n$
existeixen $n$ nombres consecutius tots ells
$k$-compostos.

\item Un nombre enter $n$ no és lliure de quadrats si hi ha
algun nombre primer $p$ tal que $p^2$ divideix $n$. Doneu tres
nombres consecutius que no siguin lliures de quadrats. Demostreu per
a tot $m\ge 2$ existeixen $m$ nombres consecutius que no són lliures
de quadrats.

\item Hi ha $m$ nombres consecutius lliures de quadrats per a
a tot $m\ge 2$?

\item Per a tot $k\ge 1$, hi ha $m$ nombres consecutius que no
siguin $k$-compostos per a a tot $m\ge 2$?

\end{enumerate}

\item 
\begin{enumerate}
    \item  Demostreu que hi ha exactament 4 anells
commutatius (amb unitat) no isomorfs dos a dos amb 4 elements.
Descriviu-los.

\item Més en general, demostreu que si $p$ és un nombre
primer, hi ha exactament 4 anells commutatius (amb unitat) no
isomorfs dos a dos amb $p^2$ elements. (Indicació: Demostreu que un
anell amb $p^2$ elements, o bé és isomorf a $\Z/p^2\Z$, o bé és un
$\Z/p\Z$ espai vectorial de dimensió 2).

\item Podeu dir quants anells commutatius amb 6 elements hi
ha, mòdul isomorfia?
\end{enumerate}

\item Considerem una aplicació $d:\Z\to \Z$ tal que $d(p)=1$
per a tot nombre primer $p$, i que $d(n m)=nd(m)+md(n)$ per a qualssevol enters $n$ i $m$.

\begin{enumerate}
\item Demostreu que $d(n)=0$ si i només si $n=0$ o $\pm1$.

\item Demostreu que $d(-n)=-d(n)$.

\item Demostreu que si $n=\prod_{i=1}^s p_i^{r_i}$ és la
factorització amb producte de nombres enters, on $p_i$ són nombres
primers diferents, $r_i\ge 1$, aleshores
$$d(n)=n\sum_{i=1}^s \frac{r_i}{p_i}.$$
(i per tant $d$ està únicament determinada).

\item Sigui $p$ és un nombre primer. Definim $v_p(n)=k$ si
$p^k$ és la màxima pot\`{e}ncia de $p$ que divideix un enter
$n$. Demostreu que si $0<v_p(n)<p$, aleshores $v_p(d(n))=v_p(n)-1$,
però, en canvi, si $v_p(n)=p$, aleshores $v_p(d(n))\ge p$.

\item Demostreu que $n$ és lliure de quadrats si i nomes si
$\gcd(n,d(n))=1$.

\item Demostreu que $n=d(n)$ si i nomes si $n=\pm p^p$ per un
nombre primer $p$.

\item Demostreu que no hi ha solucions de $d(n)=a$ per a
$a=2,3$ i $11$. Trobeu totes les solucions de $d(a)=4,5$ i $6$.
\end{enumerate}

\item Considerem un nombre primer $p\ge 3$, i definim els
enters $a_k$ (depenent de $p$) que verifiquen la seg\"{u}ent igualtat de
polinomis en $x$ amb coeficients a $\Z$:
$$\prod_{j=1}^{p-1} (x-j) = \sum_{k=0}^{p-1} a_k x^k.$$

\begin{enumerate}
\item Demostreu que $p \mid a_k$ per a $1\le k\le
p-2$. (Indicació: treballeu a $\F_p$).

\item Calculeu explícitament $a_0$, i demostreu que
$$a_1=-(p-1)!\sum_{j=1}^{p-1}\frac 1j.$$

\item Demostreu que si $p\ge 5$, aleshores $p^2\mid
a_1$. (Indicació: substitu\"{\i}u $x$ per $p$ a l'equació definitoria dels
$a_k$).

\item Utilitzeu l'apartat anterior per a demostrar que si
$p\ge 5$, aleshores
$$\sum_{j=1}^{p-1} \frac 1j \equiv 0 \pmod{p^2}$$
(pensant l'igualtat a $\F_p$ o, si ho preferiu, que el numerador de
la part esquerra de l'equació és divisible per $p^2$).

\item Demostreu que si $p\ge 5$, aleshores
$$\sum_{j=1}^{p-1} \frac 1{j^2} \equiv 0 \pmod{p}.$$
\end{enumerate}

\item Considereu la funció $\varphi$ d'Euler (o sigui,
$\varphi(n)$ és igual al nombre d'elements invertibles a l'anell
$\Z/n \Z$).
\begin{enumerate}
\item Demostreu que si $k$ i $a$ són enters positius, amb
$a\ge 2$, i $m=a^k-1$, aleshores $k$ és l'ordre de $a$ mòdul $m$.

\item Demostreu que si $k$ i $a$ són enters positius, amb
$a\ge 2$, aleshores $k\mid \varphi(a^n-1)$.

\item Demostreu que si $p \mid\varphi(m)$ però $p
\nmid m$, aleshores hi ha com a mínim un nombre $q$ primer
amb $q\mid m$ i $q\equiv 1 \pmod{p}$.

\item Demostreu que per a tot nombre primer $p$, hi ha
infinits nombres primers $q$ tals que $q\equiv 1 \pmod{p}$.
\end{enumerate}

\item Diem que un polinomi irreductible i mònic $q(x)$ de
$\F_p[x]$, on $p$ és un nombre primer, és primitiu si la classe de
$x$ a $\F_p[x]/q(x)$ és primitiva (i.e. té ordre exactament $p^d-1$,
on $d$ és el grau de $q(x)$).

\begin{enumerate}
\item Demostreu que un polinomi de la forma $x^d+a\in \F_p[x]$ no
pot ser mai primitiu.

\item Demostreu que si $p=2$ i
$2^d-1$ és un primer (s'anomenen de Mersene), aleshores tot polinomi
$q(x)\in \F_2[x]$ monic i irreductible és primitiu.

\item Demostreu que el nombre de polinomis irreductibles, primitius
i mònics de grau $d>1$ a $\F_p[x]$ és exactament $\frac
{\varphi(p^d-1)}{d}$, on $\varphi$ és la funció $\varphi$ d'Euler.

\item Un trinomi primitiu és un polinomi de la forma $q(x)=x^d+ax^s+b\in
\F_p[x]$, que és irreductible i primitiu. Demostreu que si $p=2$, un
trinomi $x^d+x^s+1\in \F_2[x]$ és primitiu si i només si la
recurr\`{e}ncia $z_n=z_{n-d}+z_{n-s} \pmod{2}$ té periode $t=2^d-1$, i
és el màxim posible (el periode és $t$ si $z_{n}=z_{n+t}$ per a tot
$n$, i $t>0$ és el més petit que ho compleix).


\item Calculeu tots els polinomis primitius per a $p=2$ i $n=2,3,4$, $p=3$, $n=2,3$, i
$p=5$, $n=2$.

\end{enumerate}

\item Direm que un polinomi $p(x)\in \Q[x]$ pren valors
enters si per a tot $n\in \Z$ es té  $p(n)\in \Z$. Denotem per
$\Z\{x\}$ el conjunt de polinomis que pren valors enters.

\begin{enumerate}
\item Demostreu per a tot $n\in \Z_{\ge 1}$, el polinomi
$$b_n(x)=\frac1{n!} \prod_{i=0}^{n-1} (x-i)$$
pren valors enters. Definim $b_0(x)=1$.

\item Donat un polinomi $p(x)\in \Q[x]$, definim
$\Delta(p(x))=p(x+1)-p(x)$. Demostreu que si $p(x)\in \Z\{x\}$,
aleshores $\Delta(p(x))\in \Z\{x\}$. Demostreu que
$\Delta(b_n(x))=b_{n-1}(x)$ per a tot $n\ge 1$.

\item Demostreu que $Z\{x\}$ és un subanell de $\Q[x]$.

\item Demostreu que si un polinomi $p(x)$ és combinació lineal entera
$$p(x)=\sum_{n=0}^r a_n b_n(x)$$
amb $a_n\in \Z$ per a tot $n$, aleshores $p(x)\in \Z\{x\}$, i a més
$a_n=\Delta^{n}(p(x))(0)$.

\item Demostreu, utilitzant l'apartat anterior, que tot polinomi de
$\Z\{x\}$ és combinació lineal entera dels $p_n(x)$, i per tant que
$$\Z\{x\}=\bigoplus_{n=0}^{\infty} \Z b_n(x). $$


\end{enumerate}

\item Una funció $f:\Z_{\ge 1} \to \C$ és
multiplicativa si $f(a b)=f(a)  f(b)$ per a tot $a$ i $b$
primers entre si.

\begin{enumerate}

\item  Demostreu que la funció $d(n)$ que compta el nombre de
divisors de $n$ és multiplicativa.

\item  Demostreu que la funció $\sigma(n)=\sum_{d\mid
n} d$, suma de divisors de $n$, és multiplicativa.

\item  Demostreu que si $f$ és multiplicativa, la seg\"{u}ent
funció també ho és:
 $$\widehat{f}(n)=\sum_{d\mid n} f(d).$$

\item  Determineu explicitament $\widehat{f}$ per a $f(n)=1$
(constant igual a 1), $f(n)=\chi_1(n)$ (la funció que val 1 si $n=1$,
i $0$ si no), $f(n)=n$ (la funció identitat) i $f=\varphi(n)$ (la
funció d'Euler).

\item  Sigui $\mu(n)$ la funció definida com $\mu(n)=0$ si $n$ és divisible per algun quadrat, $\mu(n)=1$ si $n$
té exactament un nombre parell de divisors primers i $\mu(n)=-1$ si
en té un nombre senar. Demostreu que
$$f(n)=\sum_{d\mid n} \mu(d)\widehat{f}(n/d)$$
(Pista: feu-ho primer per la funció $\chi_1(n)$ que val $0$ per a tot
$n>0$, $\chi_1(1)=1$.)

\item Demostreu que si $F(x)$ és una funció de $[1,\infty]$ als reals $\R$, i
$$G(x)=\sum_{1\le d\le x} F(x/d)$$
per a tot $x\ge 1$, aleshores
 $$F(x)=\sum_{1\le d\le x} \mu(d) G(x/d)$$
per a tot $x\ge 1$.


\item  Demostreu que, per a tot $x\ge 2$, $$\left| \sum_{1\le
n\le x} \frac{\mu(n)}{n} \right| \le 1.$$ Per a demostrar-ho podeu
fer els seg\"{u}ents passos:
\begin{enumerate}
\item Proveu que $\sum_{d\le x} \mu(d) \lfloor\frac xd \rfloor=1$
per a tot $x\ge 1$ enter.
\item Proveu que $$\left| \sum_{d\le x} \mu(d) \left( \frac xd - \lfloor\frac xd
\rfloor\right) \right| \le x-1.$$
\item Dedu\"{\i}u que $$x\left| \sum_{d\le x} \frac {\mu(d)}d  \right|
\le x$$
\end{enumerate}

\end{enumerate}

\item Definim inductivament els polinomis ciclotòmics
$\Phi_n(x)$ per a $n\ge 1$ enter com $\Phi_1(x)=x-1$ i
$$\Phi_n(x)=\frac {x^n-1}{\prod_{d\mid n,\ d<n} \Phi_d(x)}$$

\begin{enumerate}

\item   Demostreu que $\Phi_{2n}(x)=\Phi_n(-x)$ si $n$ és
senar.

\item  Demostreu que $\Phi_{p^m}(x)=\Phi_p(x^{p^{m-1}})$.

\item  Demostreu que $\Phi_{n}(x)=\Phi_q(x^{n/q})$ si
$$q=\mbox{rad}(n)=\prod_{p\mid n, \mbox{\scriptsize{ $p$ primer}}}
p.$$

\item  Demostreu que si $p$ i $q$ son primers diferents,
aleshores $\Phi_{p}(x)$ i $\Phi_{pq}(x)$ tenen tots els coeficients
iguals a $0$, $1$ o $-1$.

\item  Demostreu que si $n$ és divisible com a molt per a dos
primers senars diferents, aleshores $\Phi_{n}(x)$ té tots els
coeficients iguals a $0$, $1$ o $-1$.

\item  Calculeu $\Phi_{105}(x)$ i comproveu que té algun
coefficient diferent de $0$ i $\pm 1$.

\end{enumerate}

\item Sigui $\lfloor x \rfloor$ la part entera inferior d'un
nombre real (o sigui, el nombre enter més gran més petit que $x$).

Definim, donat un nombre real, les successions $r_0=x$,
$a_0=\lfloor r_0 \rfloor$,  i per $n\ge 1$, si $r_{n-1}= a_{n-1}$,
parem la successió, i si no
$$ r_n=\frac 1{r_{n-1}-a_{n-1}} \ \  \mbox{ i } \ \  a_n=\left\lfloor r_n \right\rfloor.$$ Denotem per
$x= [a_0;a_1,a_2,a_3,\cdots]$ l'expansió de $x$ en fracció
continua.

\begin{enumerate}

\item  Demostreu que $x$ és racional si i només si existeix un
$n$ tal que $r_n=a_n$.

\item  Donat un nombre real $x=[a_0;a_1,a_2,a_3,\cdots]$,
considerem els nombres racionals
$$q_n=[a_0;a_1,a_2,a_3,\cdots,a_n]$$ per a cada $n\ge 0$. Siguin
$h_n$ i $k_n$ el numerador i denominador de $q_n$. Demostreu que
$$h_n=a_nh_{n-1}+ h_{n-2} \mbox{ i que } k_n=a_nk_{n - 1} + k_{n - 2}.$$

\item  Demostreu que
$$k_nh_{n-1}-k_{n-1}h_n=(-1)^n \mbox{ i } \frac{h_n}{k_n}-\frac{h_{n-1}}{k_{n-1}}
= \frac{(-1)^{n+1}}{k_nk_{n-1}}.$$

\item  Per a $n\ge 1$ i $x=[a_0;a_1,a_2,a_3,\cdots]$ real,
sigui $x_n=[a_n;a_{n+1},a_{n+2},a_{n+3},\cdots]$. Demostreu que per
a tot $n\ge 1$,
$$x=\frac {h_nx_{n+1}+h_{n-1}}{k_nx_{n+1}+k_{n-1}}$$

\item  Diem que $x=[a_0;a_1,a_2,a_3,\cdots]$ és una fracció
continua eventualment periòdica si existeixen $d\ge 1$ i $n_0\ge 0$
tal que $x_{n+d}=x_{n}$ per a tot $n\ge n_0$. Demostreu que per
$x=\sqrt{2}$, per $\sqrt{3}$ i per $\sqrt{5}$, la fracció continua
és eventualment periòdica.

\item  Demostreu que si la fracció continua de $x$ és periòdica
amb periode $1$ o $2$, aleshores $x$ és arrel d'un polinomi mònic de
grau com a molt $2$ amb coeficients racionals.

\item Demostreu que si la fracció continua de $x$ és
eventualment periòdica, aleshores $x$ és arrel d'un polinomi mònic
de grau com a molt $2$ amb coeficients racionals. \emph{Comentari:} El recíproc  és cert però molt més difícil de demostrar.

\end{enumerate}




\item Sigui $\lfloor x \rfloor$ la part entera inferior d'un
nombre real (o sigui, el nombre enter més gran més petit que $x$).
Donat un nombre primer $p$, i un enter $n$, sigui $v_p(n)$ la màxima
potencia de $p$ que divideix $n$.

\begin{enumerate}

\item  Demostreu que si $x_i\in \R$ per a $i=1,\dots, n$,
aleshores $$\sum_{i=1}^{n} \left\lfloor x_i\right\rfloor \le
\left\lfloor \sum_{i=1}^{n} x_i\right\rfloor.$$

\item  Demostreu que $$v_p(n!)=\sum_{i=1}^{\infty} \left\lfloor
\frac n{p^i}\right\rfloor.$$

\item  Demostreu que si $m=a_1+\dots+a_n$, amb $a_i\in \Z_{\ge
0}$, i $p$ és un nombre primer, aleshores
$$v_p(m!)\ge \sum_{i=1}^{n} v_p(a_i!).$$
Dedu\"{\i}u que $$\frac {m!}{a_1!\cdots a_n!}\in\Z_{\ge 1}.$$

\item  Demostreu que si $a\in \Z$, aleshores el nombre $\frac
1{2a-1} \binom{2a}{a}$ és enter.


\item   Demostreu que si $a$ i $b\in\Z$, aleshores el nombre
seg\"{u}ent és enter: $$ \frac{(2a)!(2b)!}{a!b!(a+b)!}$$

\item  Demostreu que
$$N!>\left( \frac {N}{e}\right)^N$$
i dedu\"{\i}u, junt amb b), que $\sum_{p\le
N} \frac{\log(p)}{p-1} > \log(N)-1$. En particular, veiem que hi ha infinits primers.
\end{enumerate}


\item Sigui $\lfloor x \rfloor$ la part entera inferior d'un
nombre real (o sigui, el nombre enter més gran més petit que $x$).
Donat un nombre primer $p$, i un enter $n$, sigui $v_p(n)$ la màxima
potencia de $p$ que divideix $n$.

\begin{enumerate}


\item  Demostreu que $$v_p(n!)=\sum_{i=1}^{\infty} \left\lfloor
\frac n{p^i}\right\rfloor.$$

\item  Demostreu que per a tot $n\ge a\ge 0$,  $$v_p(n!)\ge
v_p(a!)+v_p((n-a)!).$$

\item  Demostreu que si  $\frac{2n}3< p\le n$ és un nombre
primer, aleshores $v_p(\binom{2n}{n})=0$.

\item  Demostreu que si  $\sqrt{2n}< p\le n$ és un nombre
primer, aleshores $v_p(\binom{2n}{n})\le 1$.

\item  Demostreu que per a tot nombre enter $n$, si prenem
\[
P_n=\{p \mbox{ nombres primers } \ | \ n+2\le p \le 2n+1\}
\]
aleshores
\[
\prod_{p\in P_n} p \le \binom{2n+1}{n}< 4^n.
\]

\item  Dedu\"{\i}u de tot això que si $n$ és prou gran,
aleshores sempre hi ha un nombre primer $p$ amb $n\le p<2n$.
(Indicació: considereu $\binom{2n}{n}$).

\end{enumerate}


\item Sigui $\chi:\F_p^* \to \C^*$ un morfisme de grups.
Denotarem per $\chi(n)=\chi(\bar n)$ si $\bar n\ne 0$ és la classe
de $n$ mòdul $p$, i $\chi(n)=0$ si $\bar n=0$. L'anomenarem caràcter
de Dirichlet mòdul $p$. Diem que $\chi=1$ si $\chi(a)=1$ per a tot
$a\in \F_p^*$.

\begin{enumerate}


\item  Demostreu que $\chi(n)=\left( \frac np \right)$ és
un caràcter de Dirichlet.

\item   Si $\chi$ i $\chi'$ són caràcters de Dirichlet
mòdul $p$, demostreu que $\chi\chi'$ (definit com $(\chi\chi')(a)=
\chi(a)\chi'(a)$) també ho és.

\item  Demostreu que $\sum_{a\in \F_p} \chi(a)=0$ si i
només si $\chi\ne 1$.

\item Sigui $\psi=e^{2\pi i/p}\in \C$ una arrel primitiva
$p$-\`{e}ssima de 1. Definim la suma de Gauss de $\chi$ com
$$G(\chi)=\sum_{a\in \F_p} \chi(a) \psi^a.$$
Calculeu $G(\chi)$ si $\chi=1$, si $\chi$ és el caràcter
modul $5$ determinat dient que $\chi(2)=i=\sqrt{-1}$, i si $\chi$
és el caràcter modul $7$ determinat dient que
$\chi(3)=\omega = e^{\pi i /3}.$.

\item  Si $\chi$ i $\chi'$ són caràcters de Dirichlet
mòdul $p$, definim $$J(\chi,\chi')=\sum_{a\in \F_p}
\chi(a)\chi'(1-a).$$ Demostreu que si $\chi$, $\chi'$ i $\chi\chi'$
són $\ne 1$, aleshores
$$J(\chi,\chi')G(\chi\chi')=G(\chi)G(\chi').$$

\item Demostreu que si $\chi\ne 1$, aleshores $\| G(\chi)\|
=\sqrt{p}$ com a número complex.

\end{enumerate}



\item L'objectiu d'aquest problema és demostrar que per a
tota base $b>1$, hi ha infinits enters $n>1$ no primers que són
pseudoprimers per la base $b$. Recordem que $n$ és pseudoprimer per
a la base $b$ si $b^{n-1}\equiv 1 \pmod{n}$.

\begin{enumerate}
\item Dóna un exemple d'un pseudoprimer $n$ no primer en la base $2$. Demostra que
si $n$ és pseudoprimer no primer en la base $2$, aleshores $2^n-1$
també ho és.
\item Més en general, demostra que si  $n$ és pseudoprimer no primer en la base
$b$, i $\gcd(b-1,n)=1$, aleshores l'enter $N=(b^n-1)/(b-1)$ també ho
és, i $N>n$.
\item Aplica l'apartat anterior per a demostrar que hi ha infinits
pseudoprimers no primers en la base $b=2,3,5$.
\item Sigui $b>1$ un enter i $p$ un nombre primer tal que $p$ no
divideix $b^3-b$. Sigui $n=(b^{2p}-1)/(b^2-1)$.
\begin{enumerate}
\item Demostra que $n$ és un enter, i que no és primer.
\item Demostra que $2p$ divideix $n-1$.
\item Demostra que $n$ és pseudoprimer en la base $b$.
\end{enumerate}
\item Conclou demostrant l'objectiu.
\end{enumerate}


\item Sigui $P_n=2\cdot 3 \cdot 5 \cdots p_n$ és producte de
tots els $n$ primers nombres primers, on $p_n$ denota el en\`{e}ssim
nombre primer. L'objectiu es demostrar que el primer $p_{n+1}$ és
l'únic nombre enter positiu $m$ tal que
$$1<2^m\left(\sum_{d\mid P_n} \frac{\mu(d)}{2^d-1} - \frac 12
\right)<2,$$ on $\mu(d)$ és la funció de M\"{o}bius que val $1$ si $d$
és producte d'un nombre parell de primers diferents, $-1$ si és
producte d'un nombre senar de primers diferents, i $0$ si no (o
sigui, és divisible per algun nombre $>1$ al quadrat).

\begin{enumerate}
\item Definim la seg\"{u}ent distribució de probabilitat als enters positius
$n\ge 1$: $p(n)=2^{-n}$. Demostreu que la probabilitat $q(d)$ que un
enter a l'atzar sigui divisible per $d$ amb aquesta distribució és
$$q(d)=\sum_{n=1}^{\infty} p(nd)=\frac{1}{2^d-1}.$$
\item Demostreu ara que la probabilitat $c(m)$ que un nombre enter a
l'atzar sigui coprimer amb $m$ és
$$c(m)=\sum_{d\mid m} \frac{\mu(d)}{2^d-1},$$
utilitzant l'anomenat principi d'inclusió-exclusió.
\item Demostreu directament que
\[c(P_n)=\sum_{\gcd(m,P_n)=1} p(m)= \frac 12 + \frac{1}{2^{p_{n+1}}} +
\sum_{i\in I_n}\frac{1}{2^i},\quad\text{ on  } I_n\subset \Z_{i>p_{n+1}}.\]
\item Demostreu que $$2^{m-p_{n+1}}<2^{m}\left(c(P_n)-\frac12\right) <2^{m-p_{n+1}+1}$$
i dedu\"{\i}u-ne el resultat.
\item Doneu una fòrmula per $p_n$ utilitzant això i la funció $\lfloor\
\rfloor$ ``part entera per sota''.
\item Una altra fòrmula ``fàcil'' per $\pi(n)$, el nombre de primers
menors que $n$ és la seg\"{u}ent:
$$ \pi(n)=\sum_{j=2}^n \left( 1 + \left\lfloor
\frac{2-\sum_{i=1}^j \left( \left\lfloor\frac
{j}{i}\right\rfloor-\left\lfloor\frac {j-1}{i}\right\rfloor\right)}
{j} \right\rfloor \right)$$
\begin{enumerate}
\item Demostreu primer que
$$d(j)=\sum_{i=1}^j \left( \left\lfloor\frac
{j}{i}\right\rfloor-\left\lfloor\frac {j-1}{i}\right\rfloor\right)$$
és el nombre de divisors de $j$.
\item Demostreu ara que
\[F(j)= 1 + \left\lfloor
\frac{2-d(j)} {j} \right\rfloor =
\begin{cases}
1 & \text{si $j$ és primer},\\
0 & \text{si $j$ és compost}.
\end{cases}
\]
\item Dedu\"{\i}u la fórmula per $\pi(n)$.
\end{enumerate}

\end{enumerate}


\item  A la classe de problemes hem analitzat l'algoritme de
Karatsuba per la multiplicació d'enters. Aquest exercici estudia una
generalització d'aquest algoritme. Considerem dos nombres positius
$x$, $y$, cadascun de $kn$ bits per certs enters $k$ i $n$ (si el
nombre de bits no és un múltiple de $k$, podem afegir zeros).
Escrivim $x$ i $y$ en base $B = 2^n$ (notem que en aquesta base tant
$x$ com $y$ tindran exactament $k$ dígits), de manera que
$x=\sum_{i=0}^{k-1} y_i B^i$ i $y=\sum_{i=0}^{k-1} y_iB^i$.
Considerem els polinomis $p(t)=\sum_{i=0}^{k-1}x_it^i$ i
$q(t)=\sum_{i=0}^{k-1}y_it^i$.
  \begin{enumerate}
  \item Trobeu una relació entre els polinomis $p(t)$, $q(t)$ i el producte $xy$.
  \item Trobeu una fórmula semblant a la de l'algoritme de Karatsuba que dóni els coeficients de $p(t)q(t)$ en termes dels valors de $p$ i $q$ en els punts
    \[
      \infty, 0, \pm 1, \pm 2,\ldots \pm (k-2), k-1.
    \]
  \item En el cas $k=3$, compteu el nombre de sumes i productes d'enters de $n$ bits que cal fer en aquest cas per obtenir el producte de $x$ i $y$.
  \item Doneu un algoritme recursiu aprofitant aquesta idea (per $k=3$) i estimeu el nombre d'operacions de bit que cal per multiplicar enters de $N$ bits.
  \item A la vista del resultat obtingut, podeu conjecturar quin és el nombre d'operacions de bit aproximat que calen amb $k$ arbitrari?
  \item Compareu el resultats amb l'algoritme \emph{schoolbook} i amb l'algoritme de \emph{Karatsuba}.
  \end{enumerate}

\item  Sigui $n\ge 2$ un nombre enter. Denotem per
$$s_m(n)=\sum_{k=1}^{m} k^{n}.$$ Donat un nombre enter
$n\ge 2$, denotem $P_n$ el conjunt de primers que el divideix.
\begin{enumerate}
\item Demostreu que si $n=p$ un nombre primer, aleshores $s_{n-1}(n-1)\equiv
-1 \pmod{n}$.
\item Demostreu que si $p$ és un primer, aleshores $s_{p-1}(n-1)=\sum_{k=1}^{p-1}
k^{n-1}$ és $\equiv -1\pmod{p}$ si $p-1$ divideix a $n-1$, i és
$\equiv 0\pmod{p}$ si no.
\item Demostreu que $s_{n-1}(n-1)\equiv -1 \pmod{n}$ si i només si per a tot
$p\in P_n$ tenim que $p(p-1)$ divideix $\frac np -1$. 

(Indicació: proveu que si $p$ divideix $n$, $s_{n-1}(n-1)=\sum_{k=1}^{n-1}
k^{n-1}$ és $\equiv -\frac np\pmod{p}$ si $p-1$ divideix a $n-1$, i
és $\equiv 0\pmod{p}$ si no).
\item Trobeu 2 nombres enters no primers $n$ tals que per a tot $p\in P_n$
tenim que $p-1$ divideix $\frac np -1$.
\item Trobeu 2 nombres enters no primers $n$ tals que per a tot $p\in P_n$
tenim que $p$ divideix $\frac np -1$.
\item Demostreu que $n$ compleix que per a tot $p\in P_n$,
$p$ divideix $\frac np -1$ si i només si
$$\sum_{p\in P_n}\frac 1p -\prod_{p\in P_n}\frac 1p \in \Z_{\ge 1}.$$
\item Demostreu que un nombre senar i lliure de quadrats que
compleixi la propietat anterior ha de tenir com a mínim 9 divisors
primers diferents.
\item Proveu amb l'ajuda de l'ordinador qu\`{e} no hi ha cap nombre
senar lliure de quadrats menor que $10^{12}$ que compleixi la
propietat anterior (es sospita que no n'hi ha cap, però no es
coneix).
\end{enumerate}



\item Definim els polinomis de Bernoulli $B_{n}(x)$ com els
coeficients del desenvolupament en serie de la funció
$$\frac{ze^{zx}}{e^z-1}=\sum_{n=0}^{\infty} B_n(x) \frac{z^n}{n!},$$
i els nombres de Bernoulli $B_n=B_n(0)$.

Definim $$s_m(n)=\sum_{k=1}^{m} k^{n}.$$

\begin{enumerate}
\item Demostreu que $B_n(x)=\sum_{k=0}^n \binom{n}{k} B_k
x^{n-k}$.
\item Demostreu que
$$s_m(n)=\frac{1}{m+1}(B_{m+1}(n+1)-B_{m+1}).$$
\item Demostreu que
$\sum_{k=0}^n \binom{n+1}{k} B_k=0$.
\item Demostreu que $B_n=0$ per a tot $n$ senar.
\item Demostreu que per a tot $n$,
$$B_{2n}+\sum_{(p-1)\mid 2n} \frac 1p \in \Z.$$
\item Demostreu que per a tot nombre primer $p$, i $n$ i $m$ no
divisibles per $p-1$ però $n\equiv m \pmod{p-1}$, aleshores
$$\frac{B_n}{n}\equiv \frac{B_m}{m} \pmod{p}$$
\end{enumerate}


\item Un número enter $n$ és abundant si la suma dels seus
divisors és més gran que $2n$ (recordem que un nombre és perfecte si
la suma dels seus divisors és igual a $2n$).
\begin{enumerate}
\item Proveu que tot múltiple positiu d'un nombre abundant és abundant.
\item Proveu que tot múltiple positiu de un nombre perfecte fora d'ell mateix és abundant.
\item Comproveu que tot nombre de la forma $6m+20$, $6m+12$ i $6m+40$ per $m>1$ i
$12m+20$ per $m\ge 1$, són suma de dos nombres abundants.
\item Proveu que tot nombre parell més gran que $46$ és suma de dos
nombres abundants, veient que els casos anteriors cobreixen tots
aquests nombres parells.
\end{enumerate}

\end{enumerate}