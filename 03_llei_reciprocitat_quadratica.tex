\subsection{Residus quadràtics i el símbol de Legendre}

L'objectiu d'aquesta secció és estudiar les solucions d'equacions quadràtiques mòdul un primer $p$. Concretament, ens fixarem en l'equació $x^2\equiv a\pmod{p}$.

\begin{definition}
 Sigui $p$ un primer. Diem que un enter $a$ no divisible per $p$ és un \emph{residu quadràtic mòdul $p$} si $a$ és un quadrat mòdul $p$. Si no, direm que $a$ és un \emph{no-residu quadràtic mòdul $p$}.
\end{definition}

Observem que, si $a\equiv a'\pmod{p}$ aleshores $a$ és un residu quadràtic mòdul $p$ si i només si $a'$ ho és. Per tant, donat $p$ és fàcil donar una llista de tots els residus quadràtics mòdul $p$.

\begin{example}
L'$1$ és l'únic residu quadràtic tant mòdul $2$ com mòdul $3$.
Els residus quadràtics mòdul $5$ són l'$1$ i el $4$, perquè $1^2\equiv 1$, $2^2\equiv 4$, i $3^2\equiv (-2)^2$, $4^2 \equiv (-1)^2$.

Els residus quadràtics mòdul $7$ són $\{1,2,4\}$.

Els residus quadràtics mòdul $11$ són $\{1,3,4,5,9\}$.
\end{example}

Introduim una notació que va bé per parlar d'aquest concepte.

\begin{definition}
El \emph{símbol de Legendre} es defineix, donats un primer \textbf{senar} $p$ i un enter $a$, com
\[
\legendre{a}{p}=\begin{cases}
0 &\text{ si } p\mid a,\\
+1 &\text{ si $a$ és un residu quadràtic mòdul $p$,}\\
-1&\text{ si $a$ és un no-residu quadràtic mòdul $p$}.
\end{cases}
\]
\end{definition}

Si $a$ i $b$ són residus quadràtics, posem $a\equiv x^2\pmod p$ i $b\equiv y^2\pmod p$, aleshores és clar que $ab\equiv (xy)^2\pmod p$ i, per tant $ab$ també és un residu quadràtic. De manera semblant, si $a\equiv x^2\pmod p$ i $ab\equiv y^2\pmod p$, aleshores $b\equiv (x^{-1}y)^2\pmod p$, del que en deduïm que el producte d'un residu amb un no-residu és un no-residu. El següent lema ens diu que el producte de dos no-residus és un residu.
\begin{lemma}
Si $p$ és un primer senar, l'aplicació $\psi\colon (\Z/p\Z)^\times \to \{\pm 1\},\quad a\mapsto \legendre{a}{p}$ és un morfisme de grups exhaustiu.
\end{lemma}
\begin{proof}
Fem servir que $G = (\Z/p\Z)^\times$ és cíclic. El nucli de l'aplicació $\psi$ el formen els quadrats, un subgrup (normal) $H$ d'índex $2$. Per tant $\psi$ és la composició de
\[
G\surjects G/H \cong \{\pm 1\}.
\]
\end{proof}

\subsection{LRQ i demostració}

Durant el segle XVIII, diversos matemàtics es van preguntar si hi havia una manera senzilla de predir com es comporta $\legendre{a}{p}$ quan variava $p$. Per exemple, quan $a=5$ obtenim la Taula~\ref{taula:lrq}.
\begin{table}[ht!]
\centering
\begin{tabular}{lcccccccccccc}

\toprule
    $p$ & 7  & \textbf{11} & 13 & 17 & \textbf{19} & 23 & \textbf{29} & \textbf{31} & 37 & \textbf{41} & 43 & 47\\
    \midrule
    $\legendre{5}{p}$& -1& \textbf{1}& -1 & -1 & \textbf{1} & -1 & \textbf{1} & \textbf{1} & -1 & \textbf{1} & -1 & -1\\
    $p\bmod 5$ & 2 & \textbf{1} & 3 & 2 & \textbf{4} & 3 &\textbf{4} & \textbf{1} & 2 &\textbf{1} & 3 & 2\\
    \bottomrule
\end{tabular}
\caption{\label{taula:lrq}Taula de $\legendre{5}{p}$ per diversos primers.}

\end{table}

Fixem-nos que sembla que el símbol $\legendre{5}{p}$ només depengui de si ${p\equiv 1,4\pmod{5}}$ o no. En canvi, si fem el mateix amb $a=7$ obtenim la Taula~\ref{taula:lrq2}.
\begin{table}[ht!]
\centering

\begin{tabular}{lcccccccccccccc}
\toprule
    $p$ & 11 & 13 & 17 & \textbf{19} & 23 & \textbf{29} & \textbf{31} & \textbf{37} & 41 & 43 & \textbf{47}&\textbf{53}&\textbf{59}&61\\
\midrule
    $\legendre{7}{p}$& -1& -1& -1 & \textbf{1} & -1 & \textbf{1} & \textbf{1} & \textbf{1} & -1 & -1 & \textbf{1} & \textbf{1} & \textbf{1}&-1\\
    $p\bmod 7$ & 4 & 6 & 3 & \textbf{5} & 2& \textbf{1} & \textbf{3} & \textbf{2} & 6 & 1 & \textbf{5} & \textbf{4} & \textbf{3}&5\\
\bottomrule
\end{tabular}

\caption{\label{taula:lrq2}Taula de $\legendre{7}{p}$ per diversos primers.}

\end{table}
Ara observem que no sembla d'entrada que hi hagi una relació tan senzilla. En canvi, per $a=11$ tornem a observar el mateix comportament que per $a=5$ (quant val $\legendre{9}{p}$?). El teorema següent explica aquest fenòmen de manera molt precisa.

\begin{theorem}[Llei de Reciprocitat Quadràtica de Gauss]
\label{thm:lrq}
Siguin $p$ i $q$ dos primers senars. Aleshores
\[
    \legendre{p}{q} = (-1)^{\frac{p-1}{2}\frac{q-1}{2}}\legendre{q}{p}=\begin{cases}
    +\legendre{q}{p} & p\equiv 1\pmod 4\text{ o } q\equiv 1 \pmod{4}\\
    -\legendre{q}{p} & p\equiv 3\pmod 4\text{ i } q\equiv 3 \pmod{4}
    \end{cases}.
    \]
A més,
\[
\legendre{-1}{p} = (-1)^{\frac{p-1}{2}}=\begin{cases}
+1&p\equiv 1\pmod{4}\\
-1&p\equiv 3\pmod{4},
\end{cases}
\quad
\legendre{2}{p} = \begin{cases}
+1 & p\equiv \pm 1\pmod {8}\\
-1 & p \equiv \pm 3 \pmod{8}.
\end{cases}
\]
\end{theorem}

Fixem-nos que, si posem $p=5$, el teorema anterior ens dona que
\[
\legendre{5}{p} = \legendre{p}{5} = \begin{cases}
+1 &p\equiv 1,4\pmod 5,\\
-1 &p\equiv 2,3\pmod 5.
\end{cases}
\]
En canvi, si $p=7$, el signe $(-1)^{\frac{p-1}{2}\frac{q-1}{2}}= (-1)^{\frac{q+1}{2}}$ depèn de com sigui $q$ mòdul $4$. Com que $\legendre{q}{7}$ depèn de $q$ mòdul $7$, la quantitat $\legendre{7}{q}$ depèn de $q$ mòdul $28$. De fet,
\[
\legendre{7}{p}=\begin{cases}
1&p\equiv 1, 3, 9, 19, 25, 27\pmod{28}\\
-1&p\equiv 5, 11, 13, 15, 17, 23\pmod{28}\\
0 & p = 7.
\end{cases}
\]

La LRQ també ens permet calcular ràpidament els símbols de Legendre: per exemple, suposem que volem saber si $211$ és un quadrat mòdul $653$.

Com que $653\equiv 1\pmod 4$,
\[
\legendre{211}{653}=\legendre{653}{211},
\]
i com que $653\equiv 20\pmod{211}$,
\[
\legendre{653}{211} = \legendre{20}{211}=\legendre{4\cdot 5}{211} = \legendre{4}{211}\legendre{5}{211}=\legendre{5}{211}.
\]
Com que $5\equiv 1\pmod 4$,
\[
\legendre{5}{211}=\legendre{211}{5}=\legendre{1}{5} = 1.
\]
Concloem que $211$ és un quadrat mòdul $653$, però fixem-nos que aquest càlcul no ens permet dir quin és $x$ tal que $x^2\equiv 211\pmod{653}$ (solució: $118^2\equiv 211\pmod{653}$).

Fixem-nos que per dur a terme el càlcul anterior cal factoritzar ($20=4\cdot 5$). Per nombres molt grans això seria un problema, però hi ha una generalització del símbol de Legendre (anomenat símbol de Jacobi) que solventa aquest problema.

\subsubsection{Demostració de la LRQ}

Sigui $p$ un primer senar, i $a$ un enter no divisible per $p$. El primer resultat que ens caldrà dona una manera eficient de calcular $\legendre{a}{p}$, i ens servirà també per la demostració de la LRQ.
\begin{proposition}[Criteri d'Euler]
 \[
 \legendre a p \equiv a^{\frac{p-1}{2}}\pmod p.
 \]
\end{proposition}
\begin{proof}
Si $a\equiv x^2\pmod p$, aleshores
\[
a^{\frac{p-1}{2}} \equiv x^{p-1}\equiv 1\pmod p,
\]
pel petit teorema de Fermat. Considerem ara la factorització
\[
x^{p-1}-1 = (x^{\frac{p-1}{2}}-1)(x^{\frac{p-1}{2}}+1).
\]
El polinomi $x^{p-1}-1$ té com a molt $p-1$ arrels a $\Z/p\Z$. Com que tots els elements de $(\Z/p\Z)^\times$ en són arrel, en deduïm que té exactament $p-1$ arrels. Aquestes s'han de dividir en arrels de cadascun dels factors. Com que la meitat d'elements (els quadrats) són arrel del primer factor, l'altra meitat (els no-quadrats) han de ser arrels del segon factor, i per tant si $a$ és un no-quadrat,
\[
a^{\frac{p-1}{2}}+1\equiv0\pmod p,
\]
com voliem veure.
\end{proof}

Observem que la proposició anterior demostra la fórmula per $\legendre{-1}{p}$. Vegem ara la fórmula per $\legendre{2}{p}$.

\begin{proposition}
Si $p$ és un primer senar, aleshores
\[
\legendre 2p=\epsilon(p)=\begin{cases}
+1&p\equiv \pm 1\pmod{8},\\
-1&p\equiv \pm 3 \pmod{8}.
\end{cases}
\]
\end{proposition}
\begin{proof}
Considerem l'anell $R=\Z[\zeta]/(p)$ amb $\zeta=\zeta_8$. Pel criteri d'Euler,
\[
\legendre{2}{p}\equiv 2^{\frac{p-1}{2}}\pmod p,
\]
i per tant serà útil trobar una arrel quadrada de $2$ mòdul $p$. Fixem-nos que $\zeta^4+1=0$, o $\zeta^2+\zeta^{-2}=0$. Per tant, si escrivim $\tau=\zeta+\zeta^{-1}$, tenim
\[
\tau^2 = (\zeta+\zeta^{-1})^2= 2.
\]
Deduïm que
\[
2^{\frac{p-1}{2}}\equiv \tau^{p-1}\pmod p.
\]
Observem que, com que $R$ és de característica $p$, tenim $\tau^p\equiv \zeta^p+\zeta^{-p}\pmod p$. Tenint en compte que $\zeta^8=1$, veiem que aquesta quantitat només depèn de $p\mod 8$, i coincideix amb $\epsilon(p)\tau$.

Per tant, $\tau \legendre{2}{p} \equiv \epsilon(p) \tau \pmod p$ i, multiplicant per $\tau$, obtenim
\[
2\legendre{2}{p}\equiv 2\epsilon(p)\pmod p.
\]
Com que $p$ és senar, trobem finalment $\legendre{2}{p}=\epsilon(p)$.
\end{proof}


Donat un primer $p$, denotarem per $\zeta_p$ el nombre complex $\zeta_p=e^{\frac{2\pi i}{p}}$. Recordem una propietat coneguda de $\zeta_p$:

\begin{lemma}
Es té, per a tot $a\in \Z$,
\[
\sum_{n=0}^{p-1} \zeta_p^{an}=\begin{cases}
p& p\mid a,\\
0& p\nmid a.
\end{cases}
\]
\end{lemma}
\begin{proof}
Si $p\mid a$, aleshores $\zeta_p^a=1$ i el resultat és obvi. En cas contrari, $\zeta_p^a\neq 1$, i per tant la suma geomètrica val
\[
\sum_{n=0}^{p-1} \zeta_p^{an} = \frac{\zeta_p^{ap}-1}{\zeta_p^a-1}=0.
\]
\end{proof}

Definim una suma semblant a l'anterior, però modificant el signe d'alguns dels seus termes.
\begin{definition}
 La \emph{suma de Gauss} associada a un element $a\in (\Z/p\Z)$ és
 \[
 \gamma_a = \sum_{n=1}^{p-1}\legendre{n}{p} \zeta_p^{an}.
 \]
\end{definition}

\begin{lemma}
La suma de Gauss $\gamma_0$ val $0$.
\end{lemma}
\begin{proof}
Com que $\gamma_0=\sum_{n=1}^{p-1} \legendre{n}{p}$ i hi ha la meitat d'elements que són residus quadràtics i la meitat que no ho són, la suma dels residus és $0$.
\end{proof}

\begin{lemma}
Per a tot enter $a$, es té $\gamma_a=\legendre{a}{p}\gamma_1$.
\end{lemma}
\begin{proof}
Pel lema anterior, podem assumir que $p\nmid a$. Aleshores:
\[
\legendre a p\sum_{n=0}^{p-1} \legendre n p \zeta_{p}^{an}= \sum_{n=0}^{p-1} \legendre{an}{p}\zeta_p^{an} = \sum_{m=0}^{p-1} \legendre{m}{p}\zeta_p^m = \gamma_1,
\]
on hem fet servir que multiplicar per $a$ permuta els elements de $\Z/p\Z$. El resultat s'obté multiplicant per $\legendre ap$.
\end{proof}

La base de la demostració de la Llei de reciprocitat quadràtica és el següent resultat.
\begin{proposition}
Per tot enter $a$ coprimer amb $p$, es té que
 \[
 \gamma_a^2 = (-1)^{\frac{p-1}{2}} p.
 \]
\end{proposition}
\begin{proof}
 Pel lema anterior, podem suposar que $a=1$. Fixem-nos que, pel criteri d'Euler,
 \[
 \gamma_a\gamma_{-a}=\legendre{a}{p}\gamma_1\legendre{-a}{p}\gamma_1=\legendre{-1}{p}\gamma_1^2 = (-1)^{\frac{p-1}{2}} \gamma_1^2.
 \]
 Per tant, caldrà veure que $\gamma_a\gamma_{-a} = p$. Per fer-ho, calculem (totes les sumes recorren els enters entre $1$ i $p-1$).
 \begin{align*}
     \sum_a \gamma_a\gamma_{-a} &= \sum_{a,m,n} \legendre{n}{p}\legendre{m}{p}\zeta_p^{an-am}=\sum_{n,m} \legendre{n}{p} \legendre{m}{p} \sum_{a} \zeta^{a(n-m)} = \sum_n p \legendre{n}{p}^2 = p(p-1).
 \end{align*}
\end{proof}

\begin{proof}[Demostració del Teorema~\ref{thm:lrq}]
Treballarem a $\Z[\zeta_p]/(q)$, que és un anell de característica $q$. No hem aconseguit una arrel quadrada de $p$, sinó de $p^*=(-1)^{\frac{p-1}2} p$. Per tant, buscarem una formula per $\legendre{p^*}{q}$ fent servir el  criteri d'Euler:
\[
\legendre{p^*}{q}= (p^*)^{\frac{q-1}{2}} \equiv \gamma_1^{q-1}\pmod{q}.
\]
Calculem
\begin{align*}
\gamma_1^q &\equiv \left(\sum_n \legendre{n}{p}\zeta_p^n\right)^q \equiv \sum_n \legendre{n}{q} \zeta_p^{qn} \equiv \gamma_q\equiv \gamma_1\legendre{q}{p}\pmod{q}.
\end{align*}
Com que $\gamma_1$ és invertible a $\Z[\zeta_p]/(q)$, podem simplificar $\gamma_1$ i obtenim
\[
\legendre{p^*}{q}\equiv \legendre{q}{p}\pmod q.
\]
Com que a $R$ es pot distingir $1$ de $-1$, d'aquesta congruència en deduïm la igualtat $\legendre{p^*}{q}=\legendre{q}{p}$.
Per acabar, fixem-nos que
\[
\legendre{p^*}{q} = \legendre{-1}{q}^{\frac{p-1}{2}} \legendre pq = (-1)^{\frac{p-1}{2}\frac{q-1}{2}} \legendre pq,
\]
i per tant hem demostrat la LRQ.
\end{proof}

\subsection{El símbol de Jacobi}
Donats un enter $a$ i un \emph{enter senar positiu} $m$, definim $\legendre{a}{m}=\prod_{p^k\parallel m} \legendre{a}{p}^k$,
on en el producte de la dreta fem servir el símbol de Legendre. Observem que si $m$ és un primer senar, aleshores aquesta definició coincideix amb el símbol de Legendre.

\begin{lemma}
\begin{enumerate}
    \item Si $a\equiv b\pmod m$ aleshores $\legendre{a}{m} = \legendre{b}{m}$.
    \item $\legendre{ab}{m}= \legendre am \legendre bm$ per a qualssevol $a$, $b$ i qualsevol enter senar positiu $m$.
    \item $\legendre{a}{mn} = \legendre am \legendre an$ per a qualsevol $a$ i qualssevol enters senars positius $m$ i $n$.
\end{enumerate}
\end{lemma}
\begin{remark}
 Notem que si $\legendre{a}{m}=-1$ aleshores $a$ no és un quadrat mòdul $m$ (ja que no ho és mòdul $p$ per algun primer $p$ que divideix $m$ a una potència senar). Però en canvi, si $\legendre{a}{m}=1$ no podem deduir que $a$ sigui un quadrat mòdul $m$. Per exemple, $\legendre{2}{15} = 1$, però els quadrats mòdul $15$ són $\{1,4,6,9,10\}$.
\end{remark}

Per acabar, veiem que també es té una versió de la Llei de reciprocitat quadràtica.

 \begin{theorem}[Llei de Reciprocitat Quadràtica pel símbol de Jacobi]
\label{thm:lrq-jacobi}
Siguin $m$ i $n$ dos enters positius senars i coprimers entre si. Aleshores
\[
    \legendre{m}{n} = (-1)^{\frac{m-1}{2}\frac{n-1}{2}}\legendre{n}{m}.
    \]
A més,
\[
\legendre{-1}{m} = (-1)^{\frac{m-1}{2}}=\begin{cases}
+1&m\equiv 1\pmod{4}\\
-1&m\equiv 3\pmod{4}
\end{cases}
,\quad \legendre{2}{m} = \begin{cases}
1 & m\equiv \pm 1\pmod {8}\\
-1 & m \equiv \pm 3 \pmod{8}.
\end{cases}
\]
\end{theorem}

\begin{example}
 Suposem que volem saber si $7411$ és un quadrat mòdul $9283$. Primer hauriem de veure que els dos són primers (sí que ho són). Aleshores, com que els dos són $\equiv 3\pmod 4$ obtenim
 \[
 \legendre{7411}{9283} = -\legendre{9283}{7411}=-\legendre{1872}{7411}.
 \]
 Si només fem servir el símbol de Legendre, ara hem de factoritzar $1872=2^4\cdot 3^2\cdot 13$. En canvi, fent servir el símbol de Jacobi només hem de treure les potències de $2$:
 \begin{align*}
 &-\legendre{1872}{7411} =-\legendre{2^4}{7411}\legendre{117}{7411} = -\legendre{7411}{117}\\
 &=-\legendre{40}{117} = -\legendre{2}{117}\legendre{5}{117} = \legendre{5}{117}=\legendre{117}{5} = \legendre{2}{5}=-1.
 \end{align*}
 A més, com que el resultat és $-1$, ja sabem que $7411$ no és un quadrat mòdul $9283$, encara que no haguem comprovat que són primers.
\end{example}
 \subsection{Aplicació: arrels quadrades mòdul \texorpdfstring{$p$}{p}}
 \label{sec:arrels-quadrades}
 
 En aquesta secció ens plantegem el problema de trobar les solucions d'una equació quadràtica $ax^2+bx+c=0$ a $\F_p=\Z/p\Z$, amb $p\geq 3$ primer. De la fórmula quadràtica se'n despren que n'hi ha prou amb saber trobar l'arrel quadrada de $D=b^2-4ac$, si en té. Fent servir la llei de reciprocitat quadràtica, hem vist que podem determinar ràpidament si $D$ és un quadrat, però que no obtenim informació sobre com trobar $\delta\in \F$ tal que $\delta^2=D$.
 
 Recordem que, si $D$ és un quadrat, aleshores $D^{\frac{p-1}{2}}\equiv 1\pmod p$.
 Per tant, $D^{\frac{p+1}{2}} \equiv D\pmod p$. Observem aleshores que, si $\frac{p+1}{2}$ és parell ($\iff$ $p\equiv 3\pmod 4$), aleshores $\delta=D^{\frac{p+1}{4}}$ satisfà $\delta^2=D$.
 
 A partir d'ara, doncs, suposarem que $p\equiv 1\pmod{4}$. Veurem dos algoritmes probabilístics per trobar $\delta$.
 
 \subsubsection{Un primer algoritme}
 Considerem l'anell $R=\F_p[x]/(x^2-D)$, i escrivim $\sqrt{D}$ per la classe de $x$ a $R$. Considerem el morfisme d'anells $\varphi\colon R\to \F_p$ que envia $a+b\sqrt{D}\mapsto a+b\delta$ (fixem-nos que encara no hem trobat $\delta$, però en tot cas sabem que existeix).
 
 Ara, triem un element $z\in \F_p^\times$ a l'atzar, i calculem (fent servir exponenciació eficient) la quantitat
 \[
 (1+z\sqrt{D})^{\frac{p-1}{2}}=u+v\sqrt{D}\in R.
 \]
 Com que $\varphi(u+v\sqrt{D}) = (1+\varphi(z)\delta)^{\frac{p-1}{2}}$ i $\#\F_p^\times=(p-1)$, necessàriament $u+v\delta\in \{0,1,-1\}$. Per tant, si $v\neq 0$ (si $v=0$, triem un altre $z\in \F_p^\times$ i repetim el procés), aleshores
 \[
 \delta\in \{-u/v, (1-u)/v, (-1-u)/v\}.
 \]
 Podem provar aquestes tres possibilitats i trobarem $\delta$. 
 \begin{algo}
   \caption{Donat $D$ i un primer $p$, troba (si existeix) una arrel quadrada de $D$ a $\F_p$.}
 \begin{python}
 def troba_arrel_quadrada(D, p):
    assert legendre_symbol(D, p) == 1
    Fp = GF(p)
    if (p+1) % 4 != 0:
        return D^((p+1)//4)
    S.<x> = Fp['x']
    R.<alpha> = S.quotient(x^2-D)
    v = 0
    while v == 0:
        u, v = ((1 + Fp.random_element() * alpha)^ZZ((p-1) / 2)).list()
    root = -u / v
    if root^2 == D: return root
    vinv = 1 / v
    root += vinv
    if root^2 == D: return root
    return root - 2 * vinv
 \end{python}
\end{algo}

\subsubsection{L'algoritme de Cipolla}
Treballarem també amb un anell de la forma $R=\F_p[x]/(x^2-u)$, però l'arrel que adjuntarem no serà l'arrel de $D$, que ja sabem que és un quadrat, sino una que ens garanteixi que $R$ és un cos. Concretament, hem de trobar $t\in\F_p$ tal que $u=t^2-D$ sigui un no-quadrat a $\F_p$. La meitat de les possibles $t$ funcionaran, perquè la meitat de residus són no-quadrats. 

\begin{theorem}
Sigui $\delta=(t+\sqrt{u})^{\frac{p+1}{2}}\in R$. Aleshores:
\begin{enumerate}
    \item $\delta^2=D$, i
    \item $\delta\in\F_p$.
\end{enumerate}
\end{theorem}
\begin{proof}
Primer calculem quant val $\sqrt{u}^p$:
\[
\sqrt{u}^p = \sqrt{u}\sqrt{u}^{p-1} = \sqrt{u} u^{\frac{p-1}{2}} = -\sqrt{u},
\]
on a l'última igualtat hem fet servir el criteri d'Euler i el fet que $u$ és un no-quadrat. Aleshores, per veure $(1)$, calculem a $R$, que és de característica $p$, i recordem que $t\in\F_p$:
\[
\delta^2=(t+\sqrt{u})^{p+1} = (t+\sqrt{u}) (t^p+\sqrt{u}^p) = (t+\sqrt{u})(t-\sqrt{u})=t^2-u=D.
\]

Per veure $(2)$, observem que el polinomi $x^2-D$ té com a molt dues arrels en qualsevol cos. Hem vist que $\pm\delta$ són dues arrels quadrades, i sabem que $x^2-D$ té dues arrels a $\F_p\subset R$ (perquè estem suposant que $D$ és un quadrat a $\F_p$). Per tant, $\pm \delta$ han de ser les arrels que busquem a $\F_p$.
\end{proof}
Com que podem calcular $\delta$ amb exponenciació modular a $R$, obtenim un algoritme per trobar l'arrel quadrada de $D$. Obtenim el següent algoritme.
\begin{algo}
   \caption{Donat $D$ i un primer $p$, troba (si existeix) una arrel quadrada de $D$ a $\F_p$.}
 \begin{python}
 def troba_arrel_quadrada(D, p):
    assert legendre_symbol(D, p) == 1
    Fp = GF(p)
    u = 0
    while legendre_symbol(u, p) != -1:
        t = Fp.random_element()
        u = t^2 - D
    S.<x> = Fp['x']
    R.<w> = S.quotient(x^2 - u)
    return Fp(((t + w)^( (p+1) // 2)).lift())
 \end{python}
\end{algo}
